\documentclass[final,3p]{elsarticle}

\bibliographystyle{elsarticle-num}
\biboptions{sort&compress}

\journal{Macro-Energy Systems Workshop 2024}

\usepackage{libertine}
\usepackage{libertinust1math}


\usepackage[utf8]{inputenc}
\usepackage[T1]{fontenc}

\graphicspath{{graphics/}, {../slides/graphics}}

\DeclareGraphicsExtensions{.pdf,.jpeg,.png}

\usepackage{amsmath}
\usepackage{amsfonts}
\usepackage{amssymb}
\usepackage{float}
\usepackage[normalem]{ulem}
\usepackage{booktabs}
\usepackage{tabularx}
\usepackage{threeparttable}
\usepackage{siunitx}
% \usepackage[parfill]{parskip}

\usepackage{url}
\usepackage[colorlinks]{hyperref}
\usepackage[sort&compress,noabbrev]{cleveref}

\urlstyle{sf}
\renewcommand{\ttdefault}{\sfdefault}

\usepackage{eurosym}

\def\co{CO${}_2$} \def\el{${}_{\textrm{el}}$} \def\th{${}_{\textrm{th}}$}
\def\l{\lambda} \def\K{\kappa} \def\m{\mu} \def\G{\Gamma} \def\d{\partial}
\def\cL{\mathcal{L}} \newcommand{\ubar}[1]{\text{\b{$#1$}}}


\usepackage{fixltx2e}

\hyphenation{net-works semi-conduc-tor under-represents over-representation}

\makeatletter
\long\def\MaketitleBox{%
  \resetTitleCounters \def\baselinestretch{1}%
   \def\baselinestretch{1}%
    \noindent\Large\@title\par\vskip12pt \noindent\normalsize\elsauthors\par\vskip10pt
    \noindent\footnotesize\itshape\elsaddress\par\vskip12pt }
\makeatother


\begin{document}

\begin{frontmatter}

\title{Price Formation in 100\% Variable Renewable Electricity Systems:\\Demand Elasticity and Storage Bidding}
\end{frontmatter}

In long-term power system models that co-optimize asset capacity with dispatch,
it is often cost-effective to invest in storage when shares of wind and solar
are high. However, it is unclear how this storage should bid in real markets
where only dispatch is optimized with limited operational foresight. In this
work, we provide an answer by lifting storage constraints from the long-term
model into the objective function using Lagrangian relaxation to create
\textit{effective bids}. The shadow prices from constraints in the long-term
model dictate the storage charging and discharging bids in the short-term model
and reveal how prices form. Some of the mechanisms are known from the hydropower
industry but are mostly concerned with dispatching single hydro-electric plants
against price forecasts
\cite{lederer1984overall,Pereira1989,Rotting1992,Fosso1999,CRAMPES2019100746}.

We discuss the contention that electricity markets in 100\% variable renewable
energy system models with no fuel costs become singular and politically
challenging with many hours with prices close to 0~\euro/MWh and few hours with
exceptionally high prices to recover investment costs
\cite{mallapragadaElectricityPricing2023,jungePropertiesDeeply2022,levinEnergyStorage2023,taylorPowerSystems2015}.
We show that this only results from assuming a perfectly inelastic demand and
that the issue disappears with even a small amount of price elasticity.
Additionally, the scarcity of generation results in new price levels reflecting
the storage's willingness to pay or sell for charging/discharging.

To illustrate bidding strategies and price formation mechanisms, we build a
simple PyPSA \cite{PyPSA} capacity expansion model with a single node and no
fuel costs, allowing the expansion of onshore wind and solar capacities with
zero marginal cost, as well as battery and chemical storage systems with weather
data for 71 years taken from \cite{rdgdr321}. We analyze cases with a perfectly
inelastic demand of 100 MW, with load shedding at a VOLL of 2000 \euro/MWh, and
inverse demand curves in similar magnitude in line with observed short-term
price elasticities \cite{hirthHowAggregate2023,RePEc:zbw:esprep:272048}
(\cref{fig:price-duration}). Furthermore, we show the effect of shortened
operational foresight to four days and price formation for non-equilibrium
capacities.

We argue that the combined interaction of elastic demand willingness to pay and
storage opportunity costs is enough to stabilize prices in systems with 100\%
renewable supply, which largely recovers investment costs even with limited
operational forecast skill and small perturbations to long-term equilibrium
capacities. Hence, price duration curves from models with inelastic demand must
be interpreted carefully.

\vspace{1cm}

\begin{figure}[!ht]
	\centering
	\footnotesize\sffamily
	\includegraphics[width=0.49\textwidth]{pdc-3.pdf}
	\includegraphics[width=0.49\textwidth]{ldc-3.pdf}
	\caption{Price and load duration curves in long-term model for varying settings for price-elastic demand.}
	\label{fig:price-duration}
\end{figure}

\newpage 
{\normalsize
\bibliography{storage_bidding}
}

\end{document}