\documentclass[final,3p]{elsarticle}

\bibliographystyle{elsarticle-num}
\biboptions{sort&compress}

\journal{International Energy Workshop}

\usepackage{libertine}
\usepackage{libertinust1math}


\usepackage[utf8]{inputenc}
\usepackage[T1]{fontenc}

\graphicspath{{graphics/}, {../slides/graphics}}

\DeclareGraphicsExtensions{.pdf,.jpeg,.png}

\usepackage{amsmath}
\usepackage{amsfonts}
\usepackage{amssymb}
\usepackage{float}
\usepackage[normalem]{ulem}
\usepackage{booktabs}
\usepackage{tabularx}
\usepackage{threeparttable}
\usepackage{siunitx}
% \usepackage[parfill]{parskip}

\usepackage{url}
\usepackage[colorlinks]{hyperref}
\usepackage[sort&compress,noabbrev]{cleveref}

\urlstyle{sf}
\renewcommand{\ttdefault}{\sfdefault}

\usepackage{eurosym}

\def\co{CO${}_2$} \def\el{${}_{\textrm{el}}$} \def\th{${}_{\textrm{th}}$}
\def\l{\lambda} \def\K{\kappa} \def\m{\mu} \def\G{\Gamma} \def\d{\partial}
\def\cL{\mathcal{L}} \newcommand{\ubar}[1]{\text{\b{$#1$}}}


\usepackage{fixltx2e}

\hyphenation{net-works semi-conduc-tor under-represents over-representation}

\makeatletter
\long\def\MaketitleBox{%
  \resetTitleCounters \def\baselinestretch{1}%
   \def\baselinestretch{1}%
    \noindent\Large\@title\par\vskip12pt \noindent\normalsize\elsauthors\par\vskip10pt
    \noindent\footnotesize\itshape\elsaddress\par\vskip12pt }
\makeatother


\begin{document}

\begin{frontmatter}

	\title{Price Formation in 100\% Variable Renewable Electricity Systems:\\Demand Elasticity and Storage Bidding}

	\author[tub]{Tom~Brown}
	\author[tub]{Fabian~Neumann\corref{cor1}}
	\ead{f.neumann@tu-berlin.de}
	\author[tub]{Iegor~Riepin}

	\cortext[cor1]{Corresponding author}
	\address[tub]{Department of Digital Transformation in Energy Systems, Institute of Energy Technology, Technische Universität Berlin (TUB), Einsteinufer 25 (TA 8), 10587, Berlin, Germany}

	\fntext[label3]{Authors are listed in alphabetical order.}

  \end{frontmatter}

\section*{Introduction}

In long-term power system models that co-optimize asset capacity with dispatch,
it is often cost-effective to invest in storage when shares of wind and solar
are high. However, it is unclear how this storage should bid in real markets
where only dispatch is optimized. It is also unclear how prices form in the
long-term model. In this work, we provide the answer by lifting storage
constraints from the long-term model into the objective function using
Lagrangian relaxation to create \textit{effective bids}. The shadow prices from
constraints in the original long-term problem dictate the storage charging and
discharging bids in the short-term model, and reveal how prices form in the
long-term model.

Furthermore, we discuss the contention that electricity markets in 100\%
variable renewable energy system models with no fuel costs become singular and
politically unacceptable with many hours with prices close to 0~\euro/MWh and
few hours with exceptionally high prices to recover investment costs
\cite{jungePropertiesDeeply2022,levinEnergyStorage2023,taylorPowerSystems2015}. We show that this only results from assuming a
perfectly inelastic demand and that the issue disappears with even a small
amount of price elasticity. Hence, price duration curves from models with
inelastic demand must be interpreted with care.

In power system models with generators and load shedding or elastic demand,
prices in long-term models that co-optimize capacity with dispatch are always
the same as the prices in short-term models where you freeze the asset
capacities from the long-term model and only optimize dispatch. This is due to
the fact that the value of lost load (VOLL)  or inverse demand curve screen out
the capital costs from the long-term model, so that only the demand and marginal
costs of the generators appear in the long-term prices. This can also be
observed in systems with energy storage. The scarcity of generation,
particularly in times of low wind and solar feed-in, means there is value to
storing energy. There is a willingness to pay a certain price to charge the
storage and a willingness to sell at a certain price to discharge the storage.
Bidding strategies of storage units then result in additional price levels so
that their price arbitrage covers their investment costs. It was shown in
\cite{Brown2020} that in long-term models storage make back their costs exactly
from the market prices, in accordance with the zero-profit rule.

We show how these storage bidding strategies work in math, using
\textit{effective bids} presented in \cite{Brown2020}. Effective bids are
constraints lifted up into the objective function using Lagrangian relaxation,
and hence shown to correspond to effective bids that affect the price formation
in the model. Some of the mechanisms are known from the hydropower industry, but
much of the corresponding literature is mostly concerned with optimising
behaviour of single hydro-electric plant (respecting reservoir constraints,
inflow, spillage, etc.) against price forecasts, rather than looking at price
formation of the whole system
\cite{lederer1984overall,Pereira1989,Rotting1992,Fosso1999,CRAMPES2019100746}.

To illustrate the bidding strategies for a variety of cases, we build a simple
PyPSA (v0.25.2) \cite{PyPSA} capacity expansion model with a single node and no
fuel costs, allowing the expansion of onshore wind and solar capacities with
zero marginal cost, as well as battery and hydrogen storage systems with
individually optimised charging, discharging power and energy capacities. For
the cost and efficiency assumptions, we rely on the \textit{technology-data}
repository (v0.6.0) \cite{lisazeyenPyPSATechnologydata2023}. The optimisation
problems are solved with Gurobi (v10.0.2). The hourly wind and solar capacity
factor time series for capacity expansion are taken for Germany from 35 out of
71 available weather years included in \cite{rdgdr321}. The remaining weather
years are used for subsequent sensitivity analyses that require a sample of
weather conditions that the capacity expansion model could not have anticipated.
We analyse cases with a perfectly inelastic demand of 100 MW, with load shedding
at a VOLL of 2000 \euro/MWh, and inverse demand curves in similar magnitude in
line with observed short-term price elasticities \cite{hirthHowAggregate2023,RePEc:zbw:esprep:272048}.
Furthermore, we include sensitivity analyses regarding the effect of shortened
operational foresight to four days and pricing behaviour for non-equilibrium
capacities.

Our argument is that the combined interaction of demand willingness to pay and
storage opportunity costs is enough to stabilise prices in systems with 100\%
renewable supply, that largely recovers investment costs even with limited
operational forecast skill and small perturbations to long-term equilibrium
capacities.

\section*{Price formation without storage}\label{sec:generators}

We start with a simple welfare maximisation problem for a single year and node
with linear supply cost and an aggregate demand utility function in a long-term
equilibrium, which you might find in an introductory textbook on energy
economics:
\begin{equation}
  \max_{d_{t}, g_{s,t}, G_s}\left[\sum_{t} U(d_t) -  \sum_s c_s G_s - \sum_{s,t} o_{s} g_{s,t}\right]  \label{eq:objl}
\end{equation}
subject to
\begin{align}
  \sum d_{t} - \sum_s g_{s,t} & =  0 \hspace{0.34cm}\perp \hspace{0.34cm} \l_t \hspace{0.34cm} \forall t \label{eq:balance} \\
  -d_{t}                        & \leq 0 \hspace{0.34cm}\perp \hspace{0.34cm} \ubar{\mu}_{t} \hspace{0.34cm} \forall t  \label{eq:dem-lower}  \\
  d_{t} - D_{t}               & \leq 0 \hspace{0.34cm}\perp \hspace{0.34cm} \bar{\mu}_{t} \hspace{0.34cm} \forall t   \label{eq:dem-upper}  \\
  -g_{s,t}                        & \leq 0 \hspace{0.34cm}\perp \hspace{0.34cm} \ubar{\mu}_{s,t} \hspace{0.34cm} \forall s,t   \label{eq:gen-lower} \\
  g_{s,t} - \bar{g}_{s,t} G_s     & \leq 0 \hspace{0.34cm}\perp \hspace{0.34cm} \bar{\mu}_{s,t} \hspace{0.34cm} \forall s,t \label{eq:gen-upper}
\end{align}
Here $t$ labels time periods representing a year of hourly load and weather
conditions, $s$ labels generators, $d_{t}$ is the demand dispatch, $U(d_t)$ is
the consumer demand utility function, $D_{t}$ the exogenously given demand
volume bid, $g_{s,t}$ is the generator dispatch, $G_s$ is the generator capacity
and $\bar{g}_{s,t}\in[0,1]$ is the capacity factor, which varies in time for
variable renewable generators like wind and solar. $\l_t$ is the marginal price
of electricity, while $\bar{\mu}_{s,t}$ and $\ubar{\mu}_{s,t}$ represent shadow
prices of the generator constraints. $c_s$ represent fixed annual costs, while
$o_s$ represent variable costs, and $U'(d_t)$ the marginal utility or inverse
demand curve.

If we assume $U(d_t)=Vd_t$ with a value of lost load greater than any generator
marginal cost $V >> o_s \forall s$ and less than any generator capital cost $V
<< o_s + c_s \forall s$, then in this model the prices are either one of $o_s$,
depending on the generator $s$ setting the price (i.e. the generator running,
but still with free capacity following the merit order principle), or the value
of lost load $V$ if all generators are maxed out $g_{s,t} = G_s \forall s$.

In this case the long-term prices are identical to the prices in the short-term
model with the frozen optimal capacities from the long-term model:
\begin{equation}
  \max_{d_{t}, g_{s,t}}\left[\sum_{t} V d_{t}  - \sum_{s,t} o_{s} g_{s,t} \right]  \label{eq:objs}
\end{equation}
subject to \crefrange{eq:balance}{eq:gen-upper}.

Proof: If $D_t < \sum_s G_{s}$, then $d_t = D_t$ since $V >> o_s$. Suppose at
time $t$ that generator $s$ is the price-setting generator. By definition,
generator $s$ is both running, and has free capacity, so in both the long- and
short-term models we have $\ubar{\mu}_{s,t} = \bar{\mu}_{s,t} = 0$.
From KKT stationarity of the Lagrangian we have:
\begin{align}
  \frac{\d \cL}{\d g_{s,t}} =  -o_s + \l_t + \ubar{\mu}_{s,t} - \bar{\mu}_{s,t}  =  -o_s + \l_t = 0
\end{align}
So $\l_t = o_s$.

If $D_t > \sum_s G_{s}$, then $d_t = \sum_s g_{s,t} = \sum_s G_{s}$ and for the
demand we have
\begin{align}
  \frac{\d \cL}{\d d_{t}} =  V -\l_t + \ubar{\mu}_{t} - \bar{\mu}_{t}  =  V - \l_t = 0
\end{align}
so that $\l_t = V = o_s + \bar{\mu}_{s,t}$ for all $s$, since because the upper
limits of generation capacities are reached $\bar{\mu}_{s,t} \geq 0$.

The only complication is if $o_s + c_s < V$ for some $s$ (excluded above), in
which case it is cheaper in the long-term model to build new capacity than to
shed load; in this case the capital costs $c_s$ will appear in the prices $\l_t$
for the long-term model. To be precise, at the hour $t$ of peak demand,
$\bar{\mu}_{s,t} = c_s$ ($\bar{\mu}_{s,t'} =0$ for $t' \neq t$) and $\l_t = o_s
+ \bar{\mu}_{s,t} = o_s + c_s$. For the demand, $\bar{\mu}_{t} = V - \l_t = V -
c_s - o_s$.

If we model the electricity demand with price elasticity by using the utility
function $U(d_t) = \alpha d_t - \beta d_t^2$ with $\alpha,\beta\geq 0$ instead
of load shedding at a fixed VOLL, the resulting optimisation problem classifies
as quadratic problem (QP), and no longer as linear problem (LP). However, we
observe the computational impact of adding a quadratic term to the objective
function to be low.

Reapplying the KKT conditions will now yield that the market clearing price is
set by the inverse demand curve $U'(d_t)$ in most cases, as long as $d_t \in (0,
D_t)$: $\l_t = U'(d_t) = \alpha - 2 \beta d_t$. While $d_t \leq \sum_s G_s$, we
still have $\l_t = o_s$, where $s$ refers to the marginal generator. When $D_t >
\sum_s G_s$ and $U'(d_t) > \max\{o_s\}$, we also have $\l_t = o_s +
\bar{\mu}_{s,t}$ as above.


\section*{Effective bids from constraints}\label{sec:effective}

If we constrain the generation in the model, e.g.~with a CO$_2$ budget, then
this is equivalent to removing the constraint and altering the bidding strategy
with the equivalent CO$_2$ price. This mechanism is well-known and will useful
for later considerations.

Consider the CO$_2$ constraint:
\begin{equation}
  \sum_{s,t} e_s g_{s,t} \leq K \hspace{0.34cm}\perp \hspace{0.34cm} \mu_{\textrm{CO}_2} \label{eq:co2}
\end{equation}
where $e_s$ is the emission factor in t$_{CO_2}$/MWh\el{} for generator
$s$ and $K$ is a cap on yearly emissions in t$_{CO_2}$.
Using Lagrangian equivalence we can take the shadow price
$\mu^*_{\textrm{CO}_2}$ from the optimised problem and lift the constraint into the
objective function:
\begin{equation}
  \cdots -\sum_{s,t} o_s g_{s,t}   - \mu^*_{\textrm{CO}_2} \left(\sum_{s,t} e_s g_{s,t} - K \right) \cdots
\end{equation}

The same problem solution can thus be obtained by replacing the CO$_2$
constraint with a direct cost of CO$_2$ and making the substitution $o_s \to o_s
+ e_s \mu^*_{\textrm{CO}_2}$. This is the \textit{effective} bid, which was
formally proved using the KKT conditions in \cite{Brown2020}. The price
formation mechanism is then the same as before, except when CO$_2$-emitting
generators are price-setting, we have $\l_t = o_s + e_s \mu^*_{\textrm{CO}_2}$.

\section*{Storage with free and infinite energy capacity}\label{sec:simple}

Consider first a simplified setup where we ignore the storage energy capacity,
and just consider the power discharging and charging. A power-to-gas-to-power
storage with cheap underground cavern storage would be close to this assumption.
We add to the above model storage units $r$ with discharging dispatch
$g^{\textrm{dis}}_{r,t}$ and power capacity $G^{\textrm{dis}}_{r}$, and storing
power $g^{\textrm{sto}}_{r,t}$ and power capacity $G^{\textrm{sto}}_{r}$. The
storing efficiency is $\eta_r^{\textrm{sto}}$ and the dispatch efficiency is
$\eta_r^{\textrm{dis}}$.

Now the optimization problem becomes:
\begin{equation}
  \max_{d_{t}, g_{s,t}, G_s,g^{\textrm{dis}}_{r,t},G^{\textrm{dis}}_{r},g^{\textrm{sto}}_{r,t},G^{\textrm{sto}}_{r}}\left[\sum_{t} Vd_{t} -  \sum_s c_s G_s - \sum_{s,t} o_{s} g_{s,t} -\sum_r c^{\textrm{sto}}_r G^{\textrm{sto}}_r -\sum_r c^{\textrm{dis}}_r G^{\textrm{dis}}_r\right]  \label{eq:objsr}
\end{equation}
when we assume no marginal costs for the storage. The demand balance constraint \eqref{eq:balance} is modified to:
\begin{align}
  d_{t} - \sum_s g_{s,t} - \sum_r g^{\textrm{dis}}_{r,t} + \sum_r g^{\textrm{sto}}_{r,t}  =  0 \hspace{0.34cm}\perp \hspace{0.34cm} \l_t \hspace{0.34cm} \forall t \label{eq:balance3}
\end{align}
i.e. charging $g^{\textrm{sto}}_{r,t}$ behaves like a demand, while discharging
$g^{\textrm{dis}}_{r,t}$ behaves like a generator.

To the generation and demand constraints \eqref{eq:generation} we add the
constraints for the storage power capacities:
\begin{align}
  -g^\circ_{r,t}\leq 0 \hspace{1cm}\perp \hspace{1cm} \ubar{\mu}^\circ_{r,t} \hspace{1cm} \forall r,t  \label{eq:storlower2} \\
  g^\circ_{r,t} - G^\circ_r \leq 0 \hspace{1cm}\perp \hspace{1cm} \bar{\mu}^\circ_{r,t} \hspace{1cm} \forall r,t \label{eq:storupper2}
\end{align}
where the symbol $\circ$ runs over $\{\textrm{sto},\textrm{dis}\}$.
We must also enforce the constraints that whatever goes into the storage must
also come out over the optimization period (i.e. state of charge cyclicity):
\begin{equation}
  (\eta_r^{\textrm{dis}})^{-1} \sum_t g^{\textrm{dis}}_{r,t} - \eta_r^{\textrm{sto}}  \sum_t g^{\textrm{sto}}_{r,t}  = 0  \hspace{1cm}\perp \hspace{1cm} \lambda_{r} \hspace{1cm} \forall r \label{eq:storconstraint}
\end{equation}

This problem is equivalent to the same problem where we take the optimal shadow
prices $\lambda^*_r$, freeze them, take the constraint \eqref{eq:storconstraint}
up into the objective function and remove the constraint
\eqref{eq:storconstraint} from the problem. The new objective function is then:
\begin{equation}
  \max_{d_{t}, g_{s,t}, G_s,g^{\textrm{dis}}_{r,t},G^{\textrm{dis}}_{r},g^{\textrm{sto}}_{r,t},G^{\textrm{sto}}_{r}}\left[\sum_{t} Vd_{t} -  \sum_s c_s G_s - \sum_{s,t} o_{s} g_{s,t} -\sum_r c^{\textrm{sto}}_r G^{\textrm{sto}}_r -\sum_r c^{\textrm{dis}}_r G^{\textrm{dis}}_r -\sum_r \lambda^*_r\left[ (\eta_r^{\textrm{dis}})^{-1} \sum_t g^{\textrm{dis}}_{r,t} - \eta_r^{\textrm{sto}}  \sum_t g^{\textrm{sto}}_{r,t}\right]  \right]  \label{eq:objst}
\end{equation}
while retaining the balance constraint \eqref{eq:balance3}, generation and demand
capacity constraints \eqref{eq:generation} and the storage capacity constraints
\eqref{eq:storlower2} and \eqref{eq:storupper2}.
The equivalence between the problem with the constraint
\eqref{eq:storconstraint} and the problem with the constraint lifted into the
objective function is a standard Lagrangian approach, which can be checked via the
KKT conditions for each problem.

The new objective function can be interpreted in the following way: The
dispatching storage is behaving exactly like a generator with marginal cost $
(\eta_r^{\textrm{dis}})^{-1} \lambda^*_r $ and capacity $G^{\textrm{dis}}_{r}$,
while the charging storage is like a demand with marginal utility $
\eta_r^{\textrm{sto}} \lambda^*_r$  and bid volume $G^{\textrm{sto}}_{r}$. The
price formation will be just like a regular electricity market, and equivalent
to the short-term optimisation with frozen capacities
$G^*_s,G^{\textrm{sto},*}_{r},G^{\textrm{dis},*}_{r}$ from the long-term model.
The objective function for the short-term problem, rearranged to group demand
and supply bids, is:
\begin{equation}
  \max_{d_{t},g^{\textrm{sto}}_{r,t}, g_{s,t}, g^{\textrm{dis}}_{r,t}}\left[\sum_{t} Vd_{t} +  \sum_{r,t} \eta_r^{\textrm{sto}}\lambda^*_r g^{\textrm{sto}}_{r,t}   - \sum_{s,t} o_{s} g_{s,t}-\sum_{r,t} (\eta_r^{\textrm{dis}})^{-1}  \lambda^*_r g^{\textrm{dis}}_{r,t} \right]  \label{eq:objst2}
\end{equation}

If $\eta_r^\circ = 1$ then the bids are the same. This does not mean that there
is no arbitrage (buy low, sell high), since the electricity prices $\l_t$ are
determined only indirectly by the demand and supply bids. If there is no cost to
moving power from one hour to another hour, it makes sense to bid at the same
price, since every participants bids at their marginal cost. The market decides
whether there should be a price arbitrage benefit. This is exactly like
transmission linking two nodes. If there is no congestion, prices equalise. If
there is congestion, prices separate. Congestion would be equivalent to either
the charging or discharging power capacity being saturated.
If $\eta_r^\circ < 1$, then there is energy, and hence money, lost by
conversion, so the discharge bid is seen to be higher than the charge bid to
compensate.

\section*{Storage with finite energy capacity}

Now consider the case where the storage energy capacity $G^{\textrm{ene}}_r$ is
also co-optimized, adding to the objective function of the long-term model
$-\sum_r c^{\textrm{ene}}_r G^{\textrm{ene}}_r$. The variable for the state of
charge is $g^{\textrm{ene}}_{r,t}$ with the cyclicity constraint corresponding to
\eqref{eq:storconstraint}:
\begin{equation}
  g^{\textrm{ene}}_{r,0} = g^{\textrm{ene}}_{r,T} = g^{\textrm{ene}}_{r,0} + \sum_{t'=1}^T \left( \eta^{\textrm{sto}}_r g^{\textrm{sto}}_{r,t'} - (\eta^{\textrm{dis}}_r)^{-1} g^{\textrm{dis}}_{r,t'} \right)   \hspace{1cm}\perp \hspace{1cm} \lambda_{r} \hspace{1cm} \forall r
\end{equation}
where $g^{\textrm{ene}}_{r,0}$ is the given initial state of charge.
For the energy capacity limits, we add two extra constraints for each hour
$t$:
\begin{align}
  -g^{\textrm{ene}}_{r,t} =   -g^{\textrm{ene}}_{r,0} - \sum_{t'=1}^t \left( \eta^{\textrm{sto}}_r g^{\textrm{sto}}_{r,t'} - (\eta^{\textrm{dis}}_r)^{-1} g^{\textrm{dis}}_{r,t'} \right) \leq  0  \hspace{1cm}\perp \hspace{1cm} \ubar{\mu}^{\textrm{ene}}_{r,t} \hspace{1cm} \forall r,t \label{eq:enelower} \\
  g^{\textrm{ene}}_{r,t} - G^{\textrm{ene}}_r = g^{\textrm{ene}}_{r,0} + \sum_{t'=1}^t \left( \eta^{\textrm{sto}}_r g^{\textrm{sto}}_{r,t'} - (\eta^{\textrm{dis}}_r)^{-1} g^{\textrm{dis}}_{r,t'} \right) - G^{\textrm{ene}}_r \leq  0  \hspace{1cm}\perp \hspace{1cm} \bar{\mu}^{\textrm{ene}}_{r,t} \hspace{1cm} \forall r,t \label{eq:eneupper}
\end{align}
If we lift these constraints into the objective function, this alters the
charging demand bid (coefficient of  $g^{\textrm{sto}}_{r,t}$) to:
\begin{equation}
  \eta^{\textrm{sto}}_r \left[ \l_r + \sum_{t'=t}^T \left( \ubar{\mu}^{\textrm{ene}}_{r,t'} -\bar{\mu}^{\textrm{ene}}_{r,t'}  \right) \right]
\end{equation}
This has an interesting interpretation: If at some future point $t'$ the storage
is empty, then $\ubar{\mu}^{\textrm{ene}}_{r,t'} \geq 0$  and the bid increases;
but if at some time $t'$ the storage is full, then
$\bar{\mu}^{\textrm{ene}}_{r,t'} \geq 0$ and bid decreases.
Similarly for the discharging dispatch bid (coefficient of
$-g^{\textrm{dis}}_{r,t}$, note minus sign of $o_sg_{s,t}$ in objective
function) to:
\begin{equation}
  (\eta^{\textrm{dis}}_r)^{-1}\left[ \l_r + \sum_{t'=t}^T \left(\ubar{\mu}^{\textrm{ene}}_{r,t'} - \bar{\mu}^{\textrm{ene}}_{r,t'}  \right)\right]
\end{equation}
If the storage is nearly empty, this increases the price of the dispatch bid; if
it's full in future, it decreases the price. It is noteworthy that like in the
previous section, charging and discharging bids are same up to a constant.


\begin{figure}
	\footnotesize\sffamily
	\begin{tabular}{ll}
		Case A: Price set by elastic demand & Case B: Price set by fuel cell \\
		\includegraphics[width=0.48\textwidth]{price-set-by-demand} &
		\includegraphics[width=0.48\textwidth]{price-set-by-discharger} \\
		Case C: Price set by electrolyser & Case D: Price set by wind and solar \\
		\includegraphics[width=0.48\textwidth]{price-set-by-charger} &
		\includegraphics[width=0.48\textwidth]{price-set-by-wind-solar} \\
	\end{tabular}
	\centering
	\includegraphics[width=0.38\textwidth]{pdc-attributed}
	\includegraphics[width=\textwidth]{../results/single/graphics/LT-country+DE-elastic+true-elastic_intercept+2000/energy_balance.pdf}
	\caption{Price duration curve, energy balance, and supply-demand curves for particular hours with market clearing. The data is shown for a scenario with wind, solar and hydrogen storage capacity expansion (no battery), price-elastic demand ($\alpha=2000, \beta=10$), and free and infinite energy capacity as seen by the singular storage bidding. }
	\label{fig:supply-demand}
\end{figure}


\section*{Brief Demonstration}



\Cref{fig:supply-demand} illustrates the supply-demand curves for specific hours
alongside their corresponding market clearing outcomes. This representation is
based on a scenario that includes the expansion of wind, solar, and hydrogen
storage capacity (no battery), price-elastic demand, and unlimited energy
capacity. It is possible to identify four distinct cases of price setting:
demand, renewables, fuel cell, and electrolyser bids. These four cases are also
reflected in the price duration curve.

The dependency of the price duration curve on demand elasticity is clearly
visible in \cref{fig:price-duration}. Introducing elastic demand eliminates the
typical capital costs spikes observed in models with inelastic demand, yielding
a much smoother price duration curve even in systems that do not include any
fuel costs, the extent of which is determined by the level of price elasticity.
Only around 50\% of the hours show prices at 0~\euro/MWh with elastic demand,
compared to around 10\% with perfectly inelastic demand or a VoLL.

We also observe that by including price elasticity, annual mean prices are more
stable between different weather years. For investors this means more
predictability and less risk (\cref{fig:annual-dist}).

If we take the optimised capacities from the long-term planning model and
optimised their dispatch for 35 other, previously unseen weather years, the
similarity in price and load duration curves is quite remarkable
(\cref{fig:myopic}). With a limited foresight of four days with a two-day
rolling horizon, we see higher prices and substantial load shedding if the
storage bidding strategies outlined above are not taken into account. This is
because periods of scarcity are no longer anticipated by the model.

This deficiency can be rectified by including the storage bidding strategies in
the short-term model (green line in \cref{fig:myopic}). For instance, if storage
has a value of 100 \euro{}/MWh, the electrolyser with efficiency 70\% bids 70
\euro{}/MWh for electricity, and the fuel cell with efficiency 50\% offers
electricity at 200 \euro{}/MWh. Since for carriers that are relatively cheap to
store the storage value is quite stable over time, this is a reasonable, yet
imperfect assumption resulting in duration curves more similar to those with
perfect operational foresight.

Finally, if we perturb the capacities by $\pm5\%$ so that they do not match with
the long-term equilibrium (also \cref{fig:myopic}), we do not observe prices to
collapse completely. In all cases, cost recovery is not perfect but acceptable
given limited operational foresight in unknown weather years and small
deviations from the long-term equilibrium.


\begin{figure}
	\centering
	\footnotesize\sffamily
	\includegraphics[width=0.49\textwidth]{pdc-3.pdf}
	\includegraphics[width=0.49\textwidth]{ldc-3.pdf}
	\caption{Price and load duration curves in long-term model for varying settings for price-elastic demand (``elastic-$\alpha$'' with $\beta=0.05\alpha$) for scenario including storage cost and battery storage.}
	\label{fig:price-duration}
\end{figure}

\begin{figure}
	\centering
	\footnotesize\sffamily
	\includegraphics[width=\textwidth]{annual-prices.pdf}
	\caption{Distribution of annual mean electricity prices for different demand elasticity scenarios.}
	\label{fig:annual-dist}
\end{figure}


\begin{figure}
	\centering
	\footnotesize\sffamily
	\includegraphics[width=0.49\textwidth]{pdc-myopic-5.pdf}
	\includegraphics[width=0.49\textwidth]{ldc-myopic-5.pdf}
	\caption{Price and load duration curves in short-term model for varying settings of operational foresight and capacity perturbations with price-elastic demand (``elastic-$\alpha$'' with $\beta=0.05\alpha$) including storage cost and battery storage.}
	\label{fig:myopic}
\end{figure}

{\normalsize
\bibliography{storage_bidding}
}

\end{document}
