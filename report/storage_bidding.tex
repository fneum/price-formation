\documentclass[final,3p]{elsarticle}


\bibliographystyle{elsarticle-num}
\biboptions{sort&compress}

\usepackage{libertine}
\usepackage{libertinust1math}

\usepackage[textsize=footnotesize]{todonotes}

\usepackage[utf8]{inputenc}
\usepackage[T1]{fontenc}

\graphicspath{{graphics/}}

\DeclareGraphicsExtensions{.pdf,.jpeg,.png}

\usepackage{amsmath}
\usepackage{amsfonts}
\usepackage{amssymb}
\usepackage{float}
\usepackage[normalem]{ulem}
\usepackage{booktabs}
\usepackage{tabularx}
\usepackage{threeparttable}
\usepackage{siunitx}

\usepackage{url}
\usepackage[colorlinks]{hyperref}
\usepackage[nameinlink,sort&compress,capitalise,noabbrev]{cleveref}

\urlstyle{sf}
\renewcommand{\ttdefault}{\sfdefault}

\usepackage[gen]{eurosym}

\def\co{CO${}_2$} \def\el{${}_{\textrm{el}}$} \def\th{${}_{\textrm{th}}$}
\def\l{\lambda} \def\K{\kappa} \def\m{\mu} \def\G{\Gamma} \def\d{\partial}
\def\cL{\mathcal{L}} \newcommand{\ubar}[1]{\text{\b{$#1$}}}


\usepackage{fixltx2e}

\hyphenation{net-works semi-conduc-tor under-represents over-representation}

\journal{ECEMP}

\begin{document}
\begin{frontmatter}

  \title{Price Formation in 100\% Variable Renewable Electricity Systems}

  \author[tub]{Tom~Brown\corref{cor1}}
  \ead{t.brown@tu-berlin.de}
  \author[tub]{Fabian~Neumann}
  \author[tub]{Iegor~Riepin}

  \cortext[cor1]{Corresponding author}
  \address[tub]{Department of Digital Transformation in Energy Systems, Institute of Energy Technology, Technische Universität Berlin (TUB), Einsteinufer 25 (TA 8), 10587, Berlin, Germany}


  \begin{abstract}
    In long-term power system models that co-optimize asset capacity with
    dispatch, it is often cost-effective to invest in storage when shares of
    wind and solar are high. However, it is unclear how this storage should bid
    in real markets where only dispatch is optimized. It is also unclear how
    prices form in the long-term model. In this note we provide the answer by
    lifting storage constraints from the long-term model into the objective
    function using Lagrangian relaxation to create ``effective bids''. The
    shadow prices from constraints in the original long-term problem dictate the
    storage charging and discharging bids in the short-term model, and reveal
    how prices form in the long-term model.
  \end{abstract}

\end{frontmatter}

\section{Introduction}

In power system models with generators and load-shedding but no storage, prices
in long-term models that co-optimize capacity with dispatch are always the same
as the prices in short-term models where you freeze the asset capacities from
the long-term model and only optimize dispatch. This is due to the fact that the
value of lost load (VOLL) screens out the capital costs from the long-term
model, so that only the VOLL and marginal costs of the generators appear in the
long-term prices. This is proved in Section \ref{sec:generators}.

This does NOT happen with storage in the model. \todo{it does!}

It was shown in \cite{Brown2020} that in long-term models storage make back
their costs exactly from the market prices (zero-profit rule). The proof is
reproduced in \ref{sec:storage}.

But how exactly are prices formed given that not all prices correspond to
marginal costs of generators anymore? What do the storage units bid?

And how does this work in short-term models? They must choose a bidding strategy
so that their price arbitrage covers their investment costs.

We show how this works in math, using effective bids introduced in
\cite{Brown2020} (recapped in Section \ref{sec:effective}). Effective bids are
constraints lifted up into the objective function using Lagrangian relaxation,
and hence shown to corespond to effective bids that affect the price formation
in the model.

NB: Conventional generators also use stored fuels (coal, gas, uranium, etc.),
just we ignore storage constraints / combine them into delivery costs for fuel.

\subsection{Literature review}

Surely this is already done in long-term hydro planning context, e.g.
opportunity cost for storage dams in Scandinavia and the Alps? Much of hydro
literature is mostly concerned with optimising behaviour of single hydro plant
(respecting reservoir constraints, inflow, spillage, etc.) against price
forecasts, see e.g. \cite{Pereira1989,Rotting1992,Fosso1999}, rather than
looking at price formation of the whole system. TODO: look at
\cite{horsley2016}. \cite{Steeger2017} seems to be doing something with the
Lagrange multipliers of reservoir constraints, but focuses on case with single
price-maker hydro plant (we assume every agent is a price-taker).

The Imperial College group around Richard Green and Iain Staffell looks at
prices with storage in a few papers, but it's all doing experimental
simulations, no analytics.

\cite{Green2015} looks at changing price duration curves in long-term equilibria
as storage is introduced, but doesn't do anything analytic. See also
\href{https://spiral.imperial.ac.uk/handle/10044/1/51138}{Green's slides}.


\cite{Staffell2016} looks at how unit commitment interfers with the merit order
curve in the absence of storage, altering prices, but no analytics (shadow
prices are not well-defined in MILPs anyway).

\cite{Ward2018} develops an algorithm for storage bidding that doesn't use
optimisation, but looks at clusters of high and low prices from historical price
duration curves.

TODO: Mine \cite{Ward2018} references, e.g.
\url{https://doi.org/10.1016/j.energy.2011.05.041},
\url{https://doi.org/10.1016/j.apenergy.2015.09.006}

\cite{Ward2019} looks model price variability and how this affects arbitrage
earnings by storage.

TODO: Strbac et al 2012: Strbac, G., M. Aunedi, D. Pudjianto, P. Djapic, et al.
(2012). Strategic Assessment of the Role and Value of Energy Storage Systems in
the UK Low Carbon Energy Future, Report for the Carbon Trust.

\cite{CRAMPES2019100746} does same KKT procedure as below for a price-taking
storage, and gets similar rules for bid prices. However, they don't consider the
long-term equilibrium and price formation in such models. They mention similar
results from the hydroelectricity literature \cite{lederer1984overall} (just
does optimization problem, no KKT). \cite{CRAMPES2019100746} provide
screening-curve-like things for the storage max-hours, which are like 2-d slices
through LCOS heatmaps in \cite{schmidtmelchior2019}.

TODO: read
\href{https://www.uio.no/studier/emner/sv/oekonomi/ECON4930/v11/undervisningsmateriale/Hydropower%20economics4.pdf}{Hydro
  economics} (2011 or later). Has good literature references (many already
caught above). This actually says all the relevant things, e.g. ``price changes
due to the basic reservoir constraints becoming active.  There may be several
price changes during a yearly cycle, depending on demand and inflow
conditions.'' and ``there are three conditions for price-setting regimes, the
reservoir remaining within its limits of empty and full, the reservoir running
empty and the reservoir running full''.

\section{Warm-up: Price formation with generators only}\label{sec:generators}

We recap this textbook stuff to prepare for the introduction of storage.

We maximize yearly social welfare for a single node with linear supply cost and
demand utility functions in a long-term equilibrium:
\begin{equation}
  \max_{d_{a,t}, g_{s,t}, G_s}\left[\sum_{a,t} u_{a,t} d_{a,t} -  \sum_s c_s G_s - \sum_{s,t} o_{s} g_{s,t}\right]  \label{eq:objl}
\end{equation}
subject to
\begin{align}
  \sum_a d_{a,t} - \sum_s g_{s,t} & =  0 \hspace{0.34cm}\perp \hspace{0.34cm} \l_t \hspace{0.34cm} \forall t \label{eq:balance} \\
  -d_{a,t}                        & \leq 0 \hspace{0.34cm}\perp \hspace{0.34cm} \ubar{\mu}_{a,t} \hspace{0.34cm} \forall a,t    \\
  d_{a,t} - D_{a,t}               & \leq 0 \hspace{0.34cm}\perp \hspace{0.34cm} \bar{\mu}_{a,t} \hspace{0.34cm} \forall a,t     \\
  -g_{s,t}                        & \leq 0 \hspace{0.34cm}\perp \hspace{0.34cm} \ubar{\mu}_{s,t} \hspace{0.34cm} \forall s,t    \\
  g_{s,t} - \bar{g}_{s,t} G_s     & \leq 0 \hspace{0.34cm}\perp \hspace{0.34cm} \bar{\mu}_{s,t} \hspace{0.34cm} \forall s,t
\end{align}
Here $t$ labels time periods representing a year of hourly load and weather conditions,
$a$ labels consumers, $s$ labels generators, $d_{a,t}$ is the demand dispatch,
$D_{a,t}$ the demand volume bid (not optimized), $g_{s,t}$ is the generator
dispatch, $G_s$ is the generator capacity and $\bar{g}_{s,t}\in[0,1]$ is the
availability/capacity factor (which varies with time for variable renewable
generators like wind and solar). $\l_t$ is the marginal price of electricity,
while $\bar{\mu}_{s,t}$ and $\ubar{\mu}_{s,t}$ represent shadow prices of the
generator constraints. $c_s$ represent fixed annual costs, while $o_s$ represent
variable costs, and $u_{a,t}$ the marginal utility.

If we simplify to the situation of only one demand for each time with a value of
lost load greater than any generator marginal cost $u_{0,t} = V >> o_s \forall
s$ and less than any generator capital cost $V << o_s + c_s \forall s$, then in
this model the prices are either one of $o_s$, depending on the generator $s$
setting the price (i.e. the generator running, but still with free capacity), or
the value of lost load $V$ if all generators are maxed out $g_{s,t} = G_s
\forall s$.

In this case the long-term prices are identical to the prices in the short-term
model with the frozen optimal capacities from the long-term model $G_{s}^*$:
\begin{equation}
  \max_{d_{t}, g_{s,t}}\left[\sum_{t} V d_{t}  - \sum_{s,t} o_{s} g_{s,t} \right]  \label{eq:objs}
\end{equation}
subject to
\begin{align}
  d_{t} - \sum_s g_{s,t}        & =  0 \hspace{0.34cm}\perp \hspace{0.34cm} \l_t \hspace{0.34cm} \forall t \label{eq:balance2}                  \\
  -d_{t}                        & \leq 0 \hspace{0.34cm}\perp \hspace{0.34cm} \ubar{\mu}_{t} \hspace{0.34cm} \forall t  \nonumber               \\
  d_{t} - D_{t}                 & \leq 0 \hspace{0.34cm}\perp \hspace{0.34cm} \bar{\mu}_{t} \hspace{0.34cm} \forall t\nonumber                  \\
  -g_{s,t}                      & \leq 0 \hspace{0.34cm}\perp \hspace{0.34cm} \ubar{\mu}_{s,t} \hspace{0.34cm} \forall s,t \nonumber            \\
  g_{s,t} - \bar{g}_{s,t} G^*_s & \leq 0 \hspace{0.34cm}\perp \hspace{0.34cm} \bar{\mu}_{s,t} \hspace{0.34cm} \forall s,t \label{eq:generation}
\end{align}

Proof: If $D_t < \sum_s G_{s}$, then $d_t = D_t$ since $V >> o_s$. Suppose at
time $t$ that generator $s$ is the price-setting generator. By definition,
generator $s$ is both running, and has free capacity, so in both the long- and
short-term models we have $\ubar{\mu}_{s,t} = \bar{\mu}_{s,t} = 0$.

From KKT stationarity of the Lagrangian we have:
\begin{align}
  \frac{\d \cL}{\d g_{s,t}} =  -o_s + \l_t + \ubar{\mu}_{s,t} - \bar{\mu}_{s,t}  =  -o_s + \l_t = 0
\end{align}
So $\l_t = o_s$.

If $D_t > \sum_s G_{s}$, then $d_t = \sum_s g_{s,t} = \sum_s G_{s}$ and for the
demand we have
\begin{align}
  \frac{\d \cL}{\d d_{t}} =  V -\l_t + \ubar{\mu}_{t} - \bar{\mu}_{t}  =  V - \l_t = 0
\end{align}
so that $\l_t = V = o_s + \bar{\mu}_{s,t}$ for all $s$.

The only complication is if $o_s + c_s < V$ for some $s$ (excluded above), in
which case it is cheaper in the long-term model to build new capacity than to
shed load; in this case the capital costs $c_s$ will appear in the prices $\l_t$
for the long-term model. To be precise, at the hour $t$ of peak demand,
$\bar{\mu}_{s,t} = c_s$ ($\bar{\mu}_{s,t'} =0$ for $t' \neq t$) and $\l_t = o_s
+ \bar{\mu}_{s,t} = o_s + c_s$. For the demand, $\bar{\mu}_{t} = V - \l_t = V -
c_s - o_s$.


\section{Price-elastic electricity demand}

QP


\section{Effective bids from constraints on generation}\label{sec:effective}


If we constrain the generation in the model, e.g. with a CO$_2$ budget, then
this is equivalent to removing the constraint and altering the bidding strategy
with the equivalent CO$_2$ price. This is economics 101, but the mechanism is
useful for later considerations.

Consider the CO$_2$ constraint:
\begin{equation}
  \sum_{s,t} e_s g_{s,t} \leq K \hspace{0.34cm}\perp \hspace{0.34cm} \mu_{\textrm{CO}_2} \label{eq:co2}
\end{equation}
where $e_s$ is the emission factor in tonne-CO$_2$ per MWh\el{} for generator
$s$ and $K$ is a cap on yearly emissions in tonne-CO$_2$ per year.

Using Lagrangian equivalence we can take the optimal value
$\mu^*_{\textrm{CO}_2}$ from the problem and lift the constraint into the
objective function:
\begin{equation}
  \cdots -\sum_{s,t} o_s g_{s,t}   - \mu^*_{\textrm{CO}_2} \left(\sum_{s,t} e_s g_{s,t} - K \right) \cdots
\end{equation}


The same problem solution can thus be obtained by replacing the CO$_2$
constraint with a direct cost of CO$_2$ and making the substitution $o_s \to o_s
+ e_s \mu^*_{\textrm{CO}_2}$. This is the ``effective'' bid. This was formally
proved using KKT in \cite{Brown2020}.

The price formation mechanism is then the same as in Section
\ref{sec:generators}, except when CO$_2$-emitting generators are price-setting,
we have $\l_t = o_s + e_s \mu^*_{\textrm{CO}_2}$.



\section{Case of simple storage with infinite energy capacity}\label{sec:simple}

Consider first a simplified setup where we ignore the storage energy capacity,
and just consider the power discharging and charging (power-to-gas-to-power with
underground cavern storage would be close to this picture).

We add to the above model storage units $r$ with discharging dispatch
$g^{\textrm{dis}}_{r,t}$ and power capacity $G^{\textrm{dis}}_{r}$, and storing
power $g^{\textrm{sto}}_{r,t}$ and power capacity $G^{\textrm{sto}}_{r}$. The
storing efficiency is $\eta_r^{\textrm{sto}}$ and the dispatch efficiency is
$\eta_r^{\textrm{dis}}$.

Now the optimization problem becomes (we assume no marginal costs for the
storage):
\begin{equation}
  \max_{d_{t}, g_{s,t}, G_s,g^{\textrm{dis}}_{r,t},G^{\textrm{dis}}_{r},g^{\textrm{sto}}_{r,t},G^{\textrm{sto}}_{r}}\left[\sum_{t} Vd_{t} -  \sum_s c_s G_s - \sum_{s,t} o_{s} g_{s,t} -\sum_r c^{\textrm{sto}}_r G^{\textrm{sto}}_r -\sum_r c^{\textrm{dis}}_r G^{\textrm{dis}}_r\right]  \label{eq:objsr}
\end{equation}
The demand balance constraint \eqref{eq:balance2} is modified to:
\begin{align}
  d_{t} - \sum_s g_{s,t} - \sum_r g^{\textrm{dis}}_{r,t} + \sum_r g^{\textrm{sto}}_{r,t}  =  0 \hspace{0.34cm}\perp \hspace{0.34cm} \l_t \hspace{0.34cm} \forall t \label{eq:balance3}
\end{align}
i.e. charging $g^{\textrm{sto}}_{r,t}$ behaves like a demand, while discharging
$g^{\textrm{dis}}_{r,t}$ behaves like a generator.

To the generation and demand constraints \eqref{eq:generation} we add the
constraints for the storage power capacities:
\begin{align}
  -g^\circ_{r,t}\leq 0 \hspace{1cm}\perp \hspace{1cm} \ubar{\mu}^\circ_{r,t} \hspace{1cm} \forall r,t  \label{eq:storlower2} \\
  g^\circ_{r,t} - G^\circ_r \leq 0 \hspace{1cm}\perp \hspace{1cm} \bar{\mu}^\circ_{r,t} \hspace{1cm} \forall r,t \label{eq:storupper2}
\end{align}
where the symbol $\circ$ runs over $\{\textrm{sto},\textrm{dis}\}$.

We must also enforce the constraints that whatever goes into the storage must
also come out over the optimization period (i.e. state of charge cyclicity):
\begin{equation}
  (\eta_r^{\textrm{dis}})^{-1} \sum_t g^{\textrm{dis}}_{r,t} - \eta_r^{\textrm{sto}}  \sum_t g^{\textrm{sto}}_{r,t}  = 0  \hspace{1cm}\perp \hspace{1cm} \lambda_{r} \hspace{1cm} \forall r \label{eq:storconstraint}
\end{equation}

This problem is equivalent to the same problem where we take the optimal shadow
prices $\lambda^*_r$, freeze them, take the constraint \eqref{eq:storconstraint}
up into the objective function and remove the constraint
\eqref{eq:storconstraint} from the problem. The new objective function is then:
\begin{equation}
  \max_{d_{t}, g_{s,t}, G_s,g^{\textrm{dis}}_{r,t},G^{\textrm{dis}}_{r},g^{\textrm{sto}}_{r,t},G^{\textrm{sto}}_{r}}\left[\sum_{t} Vd_{t} -  \sum_s c_s G_s - \sum_{s,t} o_{s} g_{s,t} -\sum_r c^{\textrm{sto}}_r G^{\textrm{sto}}_r -\sum_r c^{\textrm{dis}}_r G^{\textrm{dis}}_r -\sum_r \lambda^*_r\left[ (\eta_r^{\textrm{dis}})^{-1} \sum_t g^{\textrm{dis}}_{r,t} - \eta_r^{\textrm{sto}}  \sum_t g^{\textrm{sto}}_{r,t}\right]  \right]  \label{eq:objst}
\end{equation}
We retain the balance constraint \eqref{eq:balance3}, generation and demand
capacity constraints \eqref{eq:generation} and the storage capacity constraints
\eqref{eq:storlower2} and \eqref{eq:storupper2}.

The equivalence between the problem with the constraint
\eqref{eq:storconstraint} and the problem with the constraint lifted into the
objective function is a standard Lagrangian move, which can be checked from the
KKT conditions for each problem.

Now the interpretation of the new objective function: The dispatching storage is
behaving exactly like a generator with marginal cost $
(\eta_r^{\textrm{dis}})^{-1} \lambda^*_r $ and capacity $G^{\textrm{dis}}_{r}$,
while the charging storage is like a demand with marginal utility $
\eta_r^{\textrm{sto}} \lambda^*_r$  and bid volume $G^{\textrm{sto}}_{r}$.

The price formation will be just like a regular market, and equivalent to the
short-term optimisation with frozen capacities
$G^*_s,G^{\textrm{sto},*}_{r},G^{\textrm{dis},*}_{r}$ from the long-term model.
The objective function for the short-term problem, rearranged to group demand
and supply bids, is:
\begin{equation}
  \max_{d_{t},g^{\textrm{sto}}_{r,t}, g_{s,t}, g^{\textrm{dis}}_{r,t}}\left[\sum_{t} Vd_{t} +  \sum_{r,t} \eta_r^{\textrm{sto}}\lambda^*_r g^{\textrm{sto}}_{r,t}   - \sum_{s,t} o_{s} g_{s,t}-\sum_{r,t} (\eta_r^{\textrm{dis}})^{-1}  \lambda^*_r g^{\textrm{dis}}_{r,t} \right]  \label{eq:objst2}
\end{equation}

If $\eta_r^\circ = 1$ then the bids are the same. This doesn't mean that there
is no arbitrage (buy low, sell high), since the electricity prices $\l_t$ are
determined only indirectly by the demand and supply bids. If there is no cost to
moving power from one hour to another hour, it makes sense to bid at the same
price, since every participants bids at their marginal cost. The market decides
whether there should be a price arbitrage benefit. This is exactly like
transmission linking two nodes. If there is no congestion, prices equalise. If
there is congestion, prices separate. Congestion would be equivalent to either
the charging or discharging power capacity being saturated.

If $\eta_r^\circ < 1$, then there is energy (and hence money) lost by
conversion, so the discharge bid is seen to be higher than the charge bid to
compensate.

\section{Full storage model}

Now consider the case where the storage energy capacity $G^{\textrm{ene}}_r$ is
also co-optimized, adding to the objective function of the long-term model
$-\sum_r c^{\textrm{ene}}_r G^{\textrm{ene}}_r$. The variable for the state of
charge is $g^{\textrm{ene}}_{r,t}$, and the original state of charge is given as
$g^{\textrm{ene}}_{r,0}$. For the bidding strategy it is easier to use the
summed expression rather than the difference equation (we assume no
self-discharge here):
\begin{equation}
  g^{\textrm{ene}}_{r,t} = g^{\textrm{ene}}_{r,0} + \sum_{t'=1}^t \left( \eta^{\textrm{sto}}_r g^{\textrm{sto}}_{r,t'} - (\eta^{\textrm{dis}}_r)^{-1} g^{\textrm{dis}}_{r,t'} \right)
\end{equation}

Add the optimization variable $G^{\textrm{ene}}_r$ to the problem for the
storage energy capacity and add $-\sum_r c^{\textrm{ene}}_r G^{\textrm{ene}}_r$
to the objective function. The cyclicity constraint is the same as constraint
\eqref{eq:storconstraint}
\begin{equation}
  g^{\textrm{ene}}_{r,0} = g^{\textrm{ene}}_{r,T} = g^{\textrm{ene}}_{r,0} + \sum_{t'=1}^T \left( \eta^{\textrm{sto}}_r g^{\textrm{sto}}_{r,t'} - (\eta^{\textrm{dis}}_r)^{-1} g^{\textrm{dis}}_{r,t'} \right)   \hspace{1cm}\perp \hspace{1cm} \lambda_{r} \hspace{1cm} \forall r
\end{equation}

For the energy capacity constraint  we add two extra constraints for each hour
$t$:
\begin{align}
  -g^{\textrm{ene}}_{r,t} =   -g^{\textrm{ene}}_{r,0} - \sum_{t'=1}^t \left( \eta^{\textrm{sto}}_r g^{\textrm{sto}}_{r,t'} - (\eta^{\textrm{dis}}_r)^{-1} g^{\textrm{dis}}_{r,t'} \right) \leq  0  \hspace{1cm}\perp \hspace{1cm} \ubar{\mu}^{\textrm{ene}}_{r,t} \hspace{1cm} \forall r,t \label{eq:enelower} \\
  g^{\textrm{ene}}_{r,t} - G^{\textrm{ene}}_r = g^{\textrm{ene}}_{r,0} + \sum_{t'=1}^t \left( \eta^{\textrm{sto}}_r g^{\textrm{sto}}_{r,t'} - (\eta^{\textrm{dis}}_r)^{-1} g^{\textrm{dis}}_{r,t'} \right) - G^{\textrm{ene}}_r \leq  0  \hspace{1cm}\perp \hspace{1cm} \bar{\mu}^{\textrm{ene}}_{r,t} \hspace{1cm} \forall r,t \label{eq:eneupper}
\end{align}


If we lift these constraints into the objective function, this alters the
charging demand bid (coefficient of  $g^{\textrm{sto}}_{r,t}$) to:
\begin{equation}
  \eta^{\textrm{sto}}_r \left[ \l_r + \sum_{t'=t}^T \left( \ubar{\mu}^{\textrm{ene}}_{r,t'} -\bar{\mu}^{\textrm{ene}}_{r,t'}  \right) \right]
\end{equation}
This has a nice interpretation: If at some future point $t'$ the storage is
empty, then $\ubar{\mu}^{\textrm{ene}}_{r,t'} \geq 0$  and the bid increases;
but if at some time $t'$ the storage is full, then
$\bar{\mu}^{\textrm{ene}}_{r,t'} \geq 0$ and bid decreases.

Similarly for the discharging dispatch bid (coefficient of
$-g^{\textrm{dis}}_{r,t}$, note minus sign of $o_sg_{s,t}$ in objective
function) to:
\begin{equation}
  (\eta^{\textrm{dis}}_r)^{-1}\left[ \l_r + \sum_{t'=t}^T \left(\ubar{\mu}^{\textrm{ene}}_{r,t'} - \bar{\mu}^{\textrm{ene}}_{r,t'}  \right)\right]
\end{equation}
If the storage is nearly empty, this increases the price of the dispatch bid; if
it's full in future, it decreases the price.

Interesting: like previous section, charging and discharging bids are same up to
a constant.

Does constraint \eqref{eq:eneupper} also alters the effective capital cost of
$G^{\textrm{ene}}_r$? No, this is just the regular stationarity for
$G^{\textrm{ene}}_r$, $ c^{\textrm{ene}}_r = \sum_t
\bar{\mu}^{\textrm{ene}}_{r,t}$.

How does the capital cost of energy storage work its way into bid prices? Bids
related to capital costs of energy storage by $\bar{\mu}^{\textrm{ene}}_{r,t}$.
What about for charging and discharging power?


\section{Demonstration in real model}

TODO

Show concrete examples of what the shadow prices and bids look like.

Start easy with renewables + gas + storage.

Then work up to renewables + battery + P2G.


\subsection{PyPSA implementation}

In PyPSA the optimization is a system cost minimization:
\begin{equation}
  \min_{g_{s,t}, G_s,g^{\circ}_{r,t},G^{\circ}_{r}},\left[ \sum_s c_s G_s + \sum_{s,t} o_{s} g_{s,t} +\sum_{\circ}\sum_r c^{\circ}_r G^{\circ}_r\right]  \label{eq:objpypsa}
\end{equation}
where $\circ\in \{\textrm{dis},\textrm{sto},\textrm{ene}\}$ runs over the
storage dispatch, charging and energy capacity.

$g^{\textrm{dis}}_{r,t}$ is the dispatch of the PyPSA Link (p0) going from the
storage bus to the electricity bus with efficiency $\eta^{\textrm{dis}}_r$.
$g^{\textrm{dis}}_{r,t}$ leaves the storage bus,
$\eta^{\textrm{dis}}_rg^{\textrm{dis}}_{r,t}$ arrives at the electricity bus.

$g^{\textrm{sto}}_{r,t}$ is the dispatch of the PyPSA Link (p0) going from the
electricity bus to the storage bus with efficiency $\eta^{\textrm{sto}}_r$.
$g^{\textrm{sto}}_{r,t}$ leaves the electricity bus,
$\eta^{\textrm{sto}}_rg^{\textrm{sto}}_{r,t}$ arrives at the storage bus.

$g^{\textrm{ene}}_{r,t}$ is the energy content of the PyPSA Store (e in PyPSA),
and $g^{\textrm{sds}}_r$ is the dispatch of the PyPSA Store (p in PyPSA).
$g^{\textrm{sds}}_r > 0$ when the Store feeds power into storage bus,
$g^{\textrm{sds}}_r < 0$ when the Store absorbs power from the storage bus.

There is a storage bus which enforces the consistency between the Store dispatch
and the Links connecting the storage bus to the electricity bus:
\begin{equation}
  g^{\textrm{sds}}_{r,t} - g^{\textrm{dis}}_{r,t} + \eta^{\textrm{sto}}_r g^{\textrm{sto}}_{r,t} = 0  \hspace{1cm}\perp \hspace{1cm} \lambda^{\textrm{sds}}_{r,t} \hspace{1cm} \forall r,t \label{eq:biddingconstraint}
\end{equation}
$\lambda^{\textrm{sds}}_{r,t}$ is the storage bus's marginal price.

For hydrogen storage, $g^{\textrm{sds}}_{r,t}$ is the dispatch of the H2 Store,
$g^{\textrm{sto}}_{r,t}$ is the electrical power demand of the H2 Electrolysis
and $g^{\textrm{dis}}_{r,t}$ is the hydrogen demand of the H2 Fuel Cell.

There is a constraint for the energy state of charge of the Store:
\begin{equation}
  \eta^{\textrm{ene}}_r g^{\textrm{ene}}_{r,t-1} - g^{\textrm{ene}}_{r,t} - g^{\textrm{sds}}_{r,t} = 0  \hspace{1cm}\perp \hspace{1cm} \lambda^{\textrm{ene}}_{r,t} \hspace{1cm} \forall r,t  \label{eq:soc3}
\end{equation}

Then there are also limits on the energy capacity:
\begin{align}
  -g^{\textrm{ene}}_{r,t}\leq 0 \hspace{1cm}\perp \hspace{1cm} \ubar{\mu}^{\textrm{ene}}_{r,t} \hspace{1cm} \forall r,t  \label{eq:storlower3} \\
  g^{\textrm{ene}}_{r,t} - G^{\textrm{ene}}_r \leq 0 \hspace{1cm}\perp \hspace{1cm} \bar{\mu}^{\textrm{ene}}_{r,t} \hspace{1cm} \forall r,t \label{eq:storupper3}
\end{align}


The demand balance constraint \eqref{eq:balance3} is modified to (note the
inclusion of the efficiency for storage dispatch):
\begin{align}
  \sum_s g_{s,t} + \sum_r \eta^{\textrm{dis}}_r g^{\textrm{dis}}_{r,t} - \sum_r g^{\textrm{sto}}_{r,t}  =  d_t \hspace{0.34cm}\perp \hspace{0.34cm} \l_t \hspace{0.34cm} \forall t \label{eq:balance4}
\end{align}
i.e. charging $g^{\textrm{sto}}_{r,t}$ behaves like a demand, while discharging
$\eta^{\textrm{dis}}_rg^{\textrm{dis}}_{r,t}$ behaves like a generator.

From stationarity we get:
\begin{align}
  \frac{\d \cL}{\d g^{\textrm{sds}}_{r,t}} = \lambda^{\textrm{ene}}_{r,t} - \lambda^{\textrm{sds}}_{r,t} = 0
\end{align}
i.e. $\lambda^{\textrm{ene}}_{r,t} = \lambda^{\textrm{sds}}_{r,t}$.

We also have from stationarity:
\begin{align}
  \frac{\d \cL}{\d g^{\textrm{ene}}_{r,t}} = \lambda^{\textrm{ene}}_{r,t} - \eta^{\textrm{ene}}_r\lambda^{\textrm{ene}}_{r,t+1} + \ubar{\mu}^{\textrm{ene}}_{r,t} -  \bar{\mu}^{\textrm{ene}}_{r,t} = 0
\end{align}
NB: This is actually $\frac{d \lambda^{\textrm{ene}}_{r,t}}{dt} =
\ubar{\mu}^{\textrm{ene}}_{r,t} - \bar{\mu}^{\textrm{ene}}_{r,t}$ in a
continuous formulation.

If we lift constraints \eqref{eq:biddingconstraint}, \eqref{eq:soc3},
\eqref{eq:storlower3} and \eqref{eq:storupper3} into the objective function we
get:
\begin{align}
  \sum_{r,t} \left[ - \lambda^{\textrm{sds}}_{r,t} \left(  g^{\textrm{sds}}_{r,t} - g^{\textrm{dis}}_{r,t} + \eta^{\textrm{sto}}_r g^{\textrm{sto}}_{r,t}\right) - \lambda^{\textrm{ene}}_{r,t} \left(\eta^{\textrm{ene}}_r g^{\textrm{ene}}_{r,t-1} - g^{\textrm{ene}}_{r,t} - g^{\textrm{sds}}_{r,t}\right)  + g^{\textrm{ene}}_{r,t}\ubar{\mu}^{\textrm{ene}}_{r,t} -  \bar{\mu}^{\textrm{ene}}_{r,t}  \left(    g^{\textrm{ene}}_{r,t} - G^{\textrm{ene}}_r \right) \right]                                        \\
  =   \sum_{r,t} \left[  g^{\textrm{sds}}_{r,t} \left(\lambda^{\textrm{ene}}_{r,t} - \lambda^{\textrm{sds}}_{r,t} \right)  +g^{\textrm{dis}}_{r,t}\lambda^{\textrm{sds}}_{r,t}  -  g^{\textrm{sto}}_{r,t} \eta^{\textrm{sto}}_r \lambda^{\textrm{sds}}_{r,t} + g^{\textrm{ene}}_{r,t}\left(\lambda^{\textrm{ene}}_{r,t} -\eta^{\textrm{ene}}_r \lambda^{\textrm{ene}}_{r,t+1}  + \ubar{\mu}^{\textrm{ene}}_{r,t} -  \bar{\mu}^{\textrm{ene}}_{r,t} \right) + \bar{\mu}^{\textrm{ene}}_{r,t} G^{\textrm{ene}}_r \right] \\
  =   \sum_{r,t} \left[ g^{\textrm{dis}}_{r,t}\lambda^{\textrm{sds}}_{r,t}  -  g^{\textrm{sto}}_{r,t} \eta^{\textrm{sto}}_r \lambda^{\textrm{sds}}_{r,t} + \bar{\mu}^{\textrm{ene}}_{r,t} G^{\textrm{ene}}_r \right]
\end{align}
The cancellations in the second line are due to the KKT stationarity first order
constraints satisfied at the optimum.

So if these constraints are frozen and lifted into the objective with the
optimal shadow prices, the storage dispatch $g^{\textrm{dis}}_{r,t}$ bids as a
generator in the electricity market with time-dependent marginal cost
$\lambda^{\textrm{sds}}_{r,t}\eta^{\textrm{dis} -1}_r $ (the factor comes from
the coefficient in the electricity market balance equation \eqref{eq:balance4})
and the storage charging $g^{\textrm{sto}}_{r,t}$ bids as a demand with marginal
price $\lambda^{\textrm{sds}}_{r,t} \eta^{\textrm{sto}}_r$, or equivalently as a
generator with negative cost $-\lambda^{\textrm{sds}}_{r,t}
\eta^{\textrm{sto}}_r$.

So imagine one unit of electricity is bought to charge at exactly the marginal
price of $\lambda^{\textrm{sds}}_{r,t} \eta^{\textrm{sto}}_r$. This creates
$\eta^{\textrm{sto}}_r$ of storage energy. This is converted back to
$\eta^{\textrm{sto}}_r\eta^{\textrm{dis}}_r$ units of electricity sold at
marginal cost of $\lambda^{\textrm{sds}}_{r,t}\eta^{\textrm{dis}-1}_r $. The
total revenue is then $\lambda^{\textrm{sds}}_{r,t} \eta^{\textrm{sto}}_r$,
which exactly covers the costs of charging. If the prices are different from
marginal costs, then storage makes a profit to cover capital costs.

The generation cost $\lambda^{\textrm{sds}}_{r,t}\eta^{\textrm{dis} -1}_r $ is
the opportunity cost of selling the limited energy during times of high
electricity prices.


\appendix


\section{Zero profit rule for long-term equilibrium with
storage}\label{sec:storage}


Suppose we add storage units $r$ with discharging dispatch
$g^{\textrm{dis}}_{r,t}$ and power capacity $G^{\textrm{dis}}_{r}$, storing
power $g^{\textrm{sto}}_{r,t}$ and capacity $G^{\textrm{sto}}_{r}$, and state of
charge $g^{\textrm{ene}}_{r,t}$ and energy capacity $G^{\textrm{ene}}_{r}$. The
efficiency from hour to hour is $\eta^{\textrm{ene}}$ (for losses due to
self-discharge), the storing efficiency is $\eta^{\textrm{sto}}$ and the
dispatch efficiency is $\eta^{\textrm{dis}}$.

We add to the objective function an additional cost term:
\begin{equation*}
  -\sum_{r,\circ} c^\circ_r G^\circ_r =  -\sum_r c^{\textrm{ene}}_r G^{\textrm{ene}}_r -\sum_r c^{\textrm{sto}}_r G^{\textrm{sto}}_r -\sum_r c^{\textrm{dis}}_r G^{\textrm{dis}}_r
\end{equation*}
where the symbol $\circ$ runs over $\{\textrm{ene},\textrm{sto},\textrm{dis}\}$.
We assume no marginal costs for the dispatch.

The demand balancing equation is modified to:
\begin{equation}
  \sum_a d_{a,t} - \sum_s g_{s,t} - \sum_r g^{\textrm{dis}}_{r,t}  + \sum_r g^{\textrm{sto}}_{r,t}  =  0 \hspace{0.34cm}\perp \hspace{0.34cm} \l_t \hspace{0.34cm} \forall t
\end{equation}
The standard capacity constraints apply:
\begin{align}
  -g^\circ_{r,t}\leq 0 \hspace{1cm}\perp \hspace{1cm} \ubar{\mu}^\circ_{r,t} \hspace{1cm} \forall r,t  \label{eq:storlower} \\
  g^\circ_{r,t} - G^\circ_r \leq 0 \hspace{1cm}\perp \hspace{1cm} \bar{\mu}^\circ_{r,t} \hspace{1cm} \forall r,t \label{eq:storupper}
\end{align}
In addition we have the constraint for the consistency of the state of charge
between hours according to how much was dispatched or stored:
\begin{equation}
  g^{\textrm{ene}}_{r,t}- \eta^{\textrm{ene}}_r g^{\textrm{ene}}_{r,t-1} - \eta^{\textrm{sto}}_r g^{\textrm{sto}}_{r,t} + (\eta^{\textrm{dis}}_r)^{-1} g^{\textrm{dis}}_{r,t}  =  0 \hspace{0.14cm}\perp \hspace{0.14cm} \l^{\textrm{ene}}_{r,t} \hspace{0.14cm} \forall r,t  \label{eq:storsoc}
\end{equation}
We assume that the state of charge is cyclic $g^{\textrm{ene}}_{r,-1} =
g^{\textrm{ene}}_{r,T-1}$.

From KKT stationarity we get:
\begin{align}
  \frac{\d \cL}{\d G^\circ_{r}} = 0            & \Rightarrow - c^\circ_r + \sum_t \bar{\m}^\circ_{r,t}  = 0                                                                                                      \\
  \frac{\d \cL}{\d g^{\textrm{dis}}_{r,t}} = 0 & \Rightarrow  \l_t + \ubar{\m}^{\textrm{dis}}_{r,t} - \bar{\m}^{\textrm{dis}}_{r,t} - (\eta^{\textrm{dis}}_r)^{-1} \l^{\textrm{ene}}_{r,t}  = 0                  \\
  \frac{\d \cL}{\d g^{\textrm{sto}}_{r,t}} = 0 & \Rightarrow  -\l_t + \ubar{\m}^{\textrm{sto}}_{r,t} - \bar{\m}^{\textrm{sto}}_{r,t} + \eta^{\textrm{sto}}_r \l^{\textrm{ene}}_{r,t}  = 0                        \\
  \frac{\d \cL}{\d g^{\textrm{ene}}_{r,t}} = 0 & \Rightarrow   \ubar{\m}^{\textrm{ene}}_{r,t} - \bar{\m}^{\textrm{ene}}_{r,t} -  \l^{\textrm{ene}}_{r,t} + \eta^{\textrm{ene}}_r \l^{\textrm{ene}}_{r,t+1}   = 0
\end{align}

The zero-profit rule for storage proceeds the usual way:
\begin{align}
  \sum_\circ c^\circ_r G^\circ_r & =  \sum_{\circ,t} G^\circ_r\bar{\m}^\circ_{r,t}  =   \sum_{\circ,t} g^\circ_{r,t}\bar{\m}^\circ_{r,t} \nonumber                                                                                                                         \\
  =                              & \sum_t \left[ \l_t g^{\textrm{dis}}_{r,t} -(\eta^{\textrm{dis}}_r)^{-1} \l^{\textrm{ene}}_{r,t}  g^{\textrm{dis}}_{r,t}
  -\l_t g^{\textrm{sto}}_{r,t} + \eta^{\textrm{sto}}_r \l^{\textrm{ene}}_{r,t} g^{\textrm{sto}}_{r,t} \right.\nonumber                                                                                                                                                     \\
                                 & \left. \hspace{.5cm}-\l^{\textrm{ene}}_{r,t}g^{\textrm{ene}}_{r,t} + \eta^{\textrm{ene}}_r \l^{\textrm{ene}}_{r,t+1}g^{\textrm{ene}}_{r,t} \right] \nonumber                                                                            \\
  =                              & \sum_t \l_t \left[ g^{\textrm{dis}}_{r,t} - g^{\textrm{sto}}_{r,t}  \right] \nonumber                                                                                                                                                   \\
                                 & + \sum_t  \l^{\textrm{ene}}_{r,t} \left[ -(\eta^{\textrm{dis}}_r)^{-1} g^{\textrm{dis}}_{r,t}+ \eta^{\textrm{sto}}_r  g^{\textrm{sto}}_{r,t} -g^{\textrm{ene}}_{r,t} + \eta^{\textrm{ene}}_r g^{\textrm{ene}}_{r,t-1} \right] \nonumber \\
  =                              & \sum_t \l_t \left[ g^{\textrm{dis}}_{r,t} - g^{\textrm{sto}}_{r,t}  \right]
\end{align}
The first equality is stationarity for $G^\circ_r$; the second is
complimentarity for constraint \eqref{eq:storupper}; the third is stationarity
for $g^\circ_{r,t}$ and complimentarity for constraint \eqref{eq:storlower}; the
fouth rearranges terms and shifts the cyclic sum over $g^{\textrm{ene}}_{r,t}$;
the final equality uses the state of charge constraint \eqref{eq:storsoc}.

The final results shows that the storage recovers its capital costs by
arbitrage, charging while prices $\l_t$ are low, and discharging while prices
are high.

The relation between market value and LCOE of generators in the system are not
affected by the introduction of storage (although the optimal capacities may
change).


\bibliography{storage_bidding}


\end{document}
