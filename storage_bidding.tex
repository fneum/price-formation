%%
%% Copyright 2007, 2008, 2009 Elsevier Ltd
%% Title should be 75 characters
%
%Decreasing market value of variable renewables a result to policy and not variability
%% This file is part of the 'Elsarticle Bundle'.
%% ---------------------------------------------
%%
%% It may be distributed under the conditions of the LaTeX Project Public
%% License, either version 1.2 of this license or (at your option) any
%% later version.  The latest version of this license is in
%%    http://www.latex-project.org/lppl.txt
%% and version 1.2 or later is part of all distributions of LaTeX
%% version 1999/12/01 or later.
%%
%% The list of all files belonging to the 'Elsarticle Bundle' is
%% given in the file `manifest.txt'.
%%

%% Template article for Elsevier's document class `elsarticle'
%% with numbered style bibliographic references
%% SP 2008/03/01

%\documentclass[sort&compress,preprint,review,3p]{elsarticle}

%% Use the option review to obtain double line spacing
%% \documentclass[authoryear,preprint,review,12pt]{elsarticle}

%% Use the options 1p,twocolumn; 3p; 3p,twocolumn; 5p; or 5p,twocolumn
%% for a journal layout:
%% \documentclass[final,1p,times]{elsarticle}
%% \documentclass[final,1p,times,twocolumn]{elsarticle}
%% \documentclass[final,3p,times]{elsarticle}
%% \documentclass[final,3p,times,twocolumn]{elsarticle}
%% \documentclass[final,5p,times]{elsarticle}
%\documentclass[final,3p,times]{elsarticle}
\documentclass[final,3p,times]{elsarticle}

\usepackage[utf8]{inputenc}
\usepackage[T1]{fontenc}


%% For including figures, graphicx.sty has been loaded in
%% elsarticle.cls. If you prefer to use the old commands
%% please give \usepackage{epsfig}
\graphicspath{{graphics/}}

\DeclareGraphicsExtensions{.pdf,.jpeg,.png}



%% The amssymb package provides various useful mathematical symbols
\usepackage{amsmath}
\usepackage{amsfonts}
\usepackage{amssymb}
%% The amsthm package provides extended theorem environments
%% \usepackage{amsthm}
\usepackage{float}

\usepackage[normalem]{ulem}

\usepackage{booktabs}
\usepackage{tabularx}
\usepackage{threeparttable}
\usepackage{siunitx}

\usepackage{url}
\usepackage[colorlinks=true, citecolor=blue, linkcolor=blue, filecolor=blue,urlcolor=blue]{hyperref}

\usepackage[gen]{eurosym}

%% The lineno packages adds line numbers. Start line numbering with
%% \begin{linenumbers}, end it with \end{linenumbers}. Or switch it on
%% for the whole article with \linenumbers.
%% \usepackage{lineno}
%\usepackage{lineno}


\newcommand{\specialcell}[2][c]{%
  \begin{tabular}[#1]{@{}l@{}}#2\end{tabular}}

%use this to add space between rows
\newcommand{\ra}[1]{\renewcommand{\arraystretch}{#1}}


\def\co{CO${}_2$}
\def\el{${}_{\textrm{el}}$}
\def\th{${}_{\textrm{th}}$}



\newcommand{\ubar}[1]{\text{\b{$#1$}}}

\def\l{\lambda}
\def\K{\kappa}
\def\m{\mu}
\def\G{\Gamma}
\def\d{\partial}
\def\cL{\mathcal{L}}


\newcommand*\rot{\rotatebox{90}}
\newcommand*\OK{\ding{51}}



\usepackage{tikz}


\usepackage[europeanresistors,americaninductors]{circuitikz}
\usepackage{adjustbox}

% *** FLOAT PACKAGES ***
%
\usepackage{fixltx2e}
% fixltx2e, the successor to the earlier fix2col.sty, was written by
% Frank Mittelbach and David Carlisle. This package corrects a few problems
% in the LaTeX2e kernel, the most notable of which is that in current
% LaTeX2e releases, the ordering of single and double column floats is not
% guaranteed to be preserved. Thus, an unpatched LaTeX2e can allow a
% single column figure to be placed prior to an earlier double column
% figure. The latest version and documentation can be found at:
% http://www.ctan.org/tex-archive/macros/latex/base/



% correct bad hyphenation here
\hyphenation{op-tical net-works semi-conduc-tor under-represents over-representation}


\journal{}

\begin{document}
%\linenumbers
\begin{frontmatter}

%% Title, authors and addresses

%% use the tnoteref command within \title for footnotes;
%% use the tnotetext command for theassociated footnote;
%% use the fnref command within \author or \address for footnotes;
%% use the fntext command for theassociated footnote;
%% use the corref command within \author for corresponding author footnotes;
%% use the cortext command for theassociated footnote;
%% use the ead command for the email address,
%% and the form \ead[url] for the home page:
%% \title{Title\tnoteref{label1}}
%% \tnotetext[label1]{}
%% \author{Name\corref{cor1}\fnref{label2}}
%% \ead{email address}
%% \ead[url]{home page}
%% \fntext[label2]{}
%% \cortext[cor1]{}
%% \address{Address\fnref{label3}}
%% \fntext[label3]{}



\title{Storage bidding strategies from long-term equilibria}


%% use optional labels to link authors explicitly to addresses:
%% \author[label1,label2]{}
%% \address[label1]{}
%% \address[label2]{}

\author[kit]{T.~Brown\corref{cor1}}
\ead{tom.brown@kit.edu}



\cortext[cor1]{Corresponding author}
\address[kit]{Institute for Automation and Applied Informatics, Karlsruhe Institute of Technology, Hermann-von-Helmholtz-Platz 1, 76344 Eggenstein-Leopoldshafen, Germany}


\begin{abstract}
  In long-term power system models that co-optimize asset capacity with
  dispatch, it is often cost-effective to invest in storage when
  shares of wind and solar are high. However, it is unclear how
  this storage should bid in real markets where only dispatch is
  optimized. It is also unclear how prices form in the long-term model.
  In this note we provide the answer by lifting storage
  constraints from the long-term model into the objective function via
  the Lagrangian to create ``effective bids''.
  The shadow prices from constraints in the
  original long-term problem dictate the storage charging and discharging bids in the short-term model, and reveal how prices form in the long-term model.
\end{abstract}


\begin{keyword}
%% keywords here, in the form: keyword \sep keyword
%Here are some suggestions:
 renewable energy policy \sep storage \sep large-scale integration of renewable power generation

%% PACS codes here, in the form: \PACS code \sep code

%% MSC codes here, in the form: \MSC code \sep code
%% or \MSC[2008] code \sep code (2000 is the default)

\end{keyword}

\end{frontmatter}

\section{Introduction}

In power system models with generators and load-shedding but no
storage, prices in long-term models that co-optimize capacity with
dispatch are always the same as the prices in short-term models where
you freeze the asset capacities from the long-term model and only
optimize dispatch. This is due to the fact that the value of lost load
(VOLL) screens out the capital costs from the long-term model, so that
only the VOLL and marginal costs of the generators appear in the
long-term prices. This is proved in Section \ref{sec:generators}.

This does NOT happen with storage in the model.

It was shown in \cite{Brown2020} that in long-term models storage make
back their costs exactly from the market prices (zero-profit rule). The proof is reproduced in \ref{sec:storage}.

But how exactly are prices formed given that not all prices
correspond to marginal costs of generators anymore? What do the storage units
bid?

And how does this work in short-term models? They must choose a bidding strategy so that their price arbitrage covers their investment costs.

We show how this works in math, using effective bids introduced in \cite{Brown2020} (recapped in Section \ref{sec:effective}). Effective bids are constraints lifted up into the objective function using Lagrangian equivalence, and hence shown to corespond to effective bids that affect the price formation in the model.

TODO Surely this is already done in long-term hydro planning context, e.g. opportunity cost for storage dams in Scandinavia and the Alps?

NB: Conventional generators also use stored fuels (coal, gas, uranium, etc.), just we ignore storage constraints / combine them into delivery costs for fuel.

\section{Warm-up: Price formation with generators only}\label{sec:generators}

We maximize yearly social welfare for a single node with linear supply cost and demand utility functions in a long-term equilibrium:
\begin{equation}
    \max_{d_{a,t}, g_{s,t}, G_s}\left[\sum_{a,t} u_{a,t} d_{a,t} -  \sum_s c_s G_s - \sum_{s,t} o_{s} g_{s,t}\right]  \label{eq:objl}
\end{equation}
subject to
\begin{align}
   \sum_a d_{a,t} - \sum_s g_{s,t} & =  0 \hspace{0.34cm}\perp \hspace{0.34cm} \l_t \hspace{0.34cm} \forall t \label{eq:balance}\\
    -d_{a,t} & \leq 0 \hspace{0.34cm}\perp \hspace{0.34cm} \ubar{\mu}_{a,t} \hspace{0.34cm} \forall a,t \\
    d_{a,t} - D_{a,t} & \leq 0 \hspace{0.34cm}\perp \hspace{0.34cm} \bar{\mu}_{a,t} \hspace{0.34cm} \forall a,t \\
    -g_{s,t} & \leq 0 \hspace{0.34cm}\perp \hspace{0.34cm} \ubar{\mu}_{s,t} \hspace{0.34cm} \forall s,t \\
         g_{s,t} - \bar{g}_{s,t} G_s & \leq 0 \hspace{0.34cm}\perp \hspace{0.34cm} \bar{\mu}_{s,t} \hspace{0.34cm} \forall s,t
\end{align}
Here $t$ labels time periods representing a year of load and weather conditions, $a$ labels consumers, $s$ labels generators, $d_{a,t}$ is the
demand dispatch, $D_{a,t}$ the demand volume bid (not optimized), $g_{s,t}$ is the generator dispatch, $G_s$ is the generator
capacity and $\bar{g}_{s,t}\in[0,1]$ is the availability/capacity factor (which
varies with time for variable renewable generators like wind and
solar). $\l_t$ is the marginal price of electricity, while
$\bar{\mu}_{s,t}$ and $\ubar{\mu}_{s,t}$ represent shadow prices of
the generator constraints. $c_s$ represent fixed annual costs, while
$o_s$ represent variable costs, and $u_{a,t}$ the marginal utility.

If we simplify to the situation of only one demand for each time with
a value of lost load greater than any generator marginal cost $u_{0,t}
= V >> o_s \forall s$ and less than any generator capital cost $V <<
c_s \forall s$, then in this model the prices are either one of $o_s$,
depending on the generator $s$ setting the price (i.e. the generator
running, but still with free capacity), or the value of lost load $V$
if all generators are maxed out $g_{s,t} = G_s \forall s$.

In this case the long-term prices are identical to the prices in the short-term model with the frozen optimal capacities from the long-term model $G_{s}^*$:
\begin{equation}
    \max_{d_{t}, g_{s,t}}\left[\sum_{t} V d_{t}  - \sum_{s,t} o_{s} g_{s,t} \right]  \label{eq:objs}
\end{equation}
subject to
\begin{align}
   d_{t} - \sum_s g_{s,t} & =  0 \hspace{0.34cm}\perp \hspace{0.34cm} \l_t \hspace{0.34cm} \forall t \label{eq:balance2}\\
    -d_{t} & \leq 0 \hspace{0.34cm}\perp \hspace{0.34cm} \ubar{\mu}_{t} \hspace{0.34cm} \forall t  \nonumber \\
    d_{t} - D_{t} & \leq 0 \hspace{0.34cm}\perp \hspace{0.34cm} \bar{\mu}_{t} \hspace{0.34cm} \forall t\nonumber  \\
    -g_{s,t} & \leq 0 \hspace{0.34cm}\perp \hspace{0.34cm} \ubar{\mu}_{s,t} \hspace{0.34cm} \forall s,t \nonumber \\
         g_{s,t} - \bar{g}_{s,t} G^*_s & \leq 0 \hspace{0.34cm}\perp \hspace{0.34cm} \bar{\mu}_{s,t} \hspace{0.34cm} \forall s,t \label{eq:generation}
\end{align}

Proof: Suppose at time $t$ that $s$ is the price-setting generator (or demand, if all generators are maxed out). By definition, generator $s$ is both running, and has free capacity, so in both the long- and short-term models we have $\ubar{\mu}_{s,t} = \bar{\mu}_{s,t} = 0$.

From KKT stationarity of the Lagrangian we have:
\begin{align}
\frac{\d \cL}{\d g_{s,t}} =  -o_s + \l_t + \ubar{\mu}_{s,t} - \bar{\mu}_{s,t}  =  -o_s + \l_t = 0
\end{align}
So $\l_t = o_s$.

The only complication is if $c_s < V$ (excluded above), in which case
it is cheaper in the long-term model to build new capacity than to
shed load; in this case the capital costs $c_s$ will appear in the
prices $\l_t$.


\section{Effective bids from constraints on generation}\label{sec:effective}


If we constrain the generation in the model, e.g. with a CO$_2$ budget, then this is equivalent to removing the constraint and altering the bidding strategy with the equivalent CO$_2$ price. This is economics 101, but the mechanism is useful for later considerations.

Consider the CO$_2$ constraint:
\begin{equation}
  \sum_{s,t} e_s g_{s,t} \leq K \hspace{0.34cm}\perp \hspace{0.34cm} \mu_{\textrm{CO}_2} \label{eq:co2}
\end{equation}
where $e_s$ is the emission factor in tonne-CO$_2$ per MWh\el{} for
generator $s$ and $K$ is a cap on yearly emissions in tonne-CO$_2$ per
year.

Using Lagrangian equivalence we can take the optimal value $\mu^*_{\textrm{CO}_2}$ from the problem and lift the constraint into the objective function:
\begin{equation}
  \cdots -\sum_{s,t} o_s g_{s,t}   - \mu^*_{\textrm{CO}_2} \left(\sum_{s,t} e_s g_{s,t} - K \right) \cdots
\end{equation}


The same problem solution can thus be obtained by replacing the CO$_2$ constraint with a direct cost of CO$_2$ and making the substitution $o_s \to o_s + e_s \mu^*_{\textrm{CO}_2}$. This is the ``effective'' bid. This was formally proved using KKT in \cite{Brown2020}.

The price formation mechanism is then the same as in Section \ref{sec:generators}, except when CO$_2$-emitting generators are price-setting, we have $\l_t = o_s + e_s \mu^*_{\textrm{CO}_2}$.



\section{Case of simple storage with infinite energy capacity}\label{sec:simple}

Consider first a simplified setup where we ignore the storage energy
capacity, and just consider the power discharging and charging
(power-to-gas-to-power with geological storage would be close to this
picture).

We add to the above model storage units $r$ with discharging dispatch $g^{\textrm{dis}}_{r,t}$ and power capacity $G^{\textrm{dis}}_{r}$, and storing power $g^{\textrm{sto}}_{r,t}$ and power capacity $G^{\textrm{sto}}_{r}$. The storing efficiency is $\eta_r^{\textrm{sto}}$ and the dispatch efficiency is $\eta_r^{\textrm{dis}}$.

Now the optimization problem becomes (we assume no marginal costs for the storage):
\begin{equation}
    \max_{d_{t}, g_{s,t}, G_s,g^{\textrm{dis}}_{r,t},G^{\textrm{dis}}_{r},g^{\textrm{sto}}_{r,t},G^{\textrm{sto}}_{r}}\left[\sum_{t} Vd_{t} -  \sum_s c_s G_s - \sum_{s,t} o_{s} g_{s,t} -\sum_r c^{\textrm{sto}}_r G^{\textrm{sto}}_r -\sum_r c^{\textrm{dis}}_r G^{\textrm{dis}}_r\right]  \label{eq:objsr}
\end{equation}
The demand balance constraint \eqref{eq:balance2} is modified to:
\begin{align}
   d_{t} - \sum_s g_{s,t} - \sum_r g^{\textrm{dis}}_{r,t} + \sum_r g^{\textrm{sto}}_{r,t}  =  0 \hspace{0.34cm}\perp \hspace{0.34cm} \l_t \hspace{0.34cm} \forall t \label{eq:balance3}
\end{align}
i.e. charging $g^{\textrm{sto}}_{r,t}$ behaves like a demand, while discharging $g^{\textrm{dis}}_{r,t}$ behaves like a generator.

To the generation and demand constraints \eqref{eq:generation} we add the constraints for the storage power capacities:
\begin{align}
    -g^\circ_{r,t}\leq 0 \hspace{1cm}\perp \hspace{1cm} \ubar{\mu}^\circ_{r,t} \hspace{1cm} \forall r,t  \label{eq:storlower2}\\
    g^\circ_{r,t} - G^\circ_r \leq 0 \hspace{1cm}\perp \hspace{1cm} \bar{\mu}^\circ_{r,t} \hspace{1cm} \forall r,t \label{eq:storupper2}
\end{align}
where the symbol $\circ$ runs over $\{\textrm{sto},\textrm{dis}\}$.

We must also enforce the constraints that whatever goes into the storage must also come out over the optimization period:
\begin{equation}
 (\eta_r^{\textrm{dis}})^{-1} \sum_t g^{\textrm{dis}}_{r,t} - \eta_r^{\textrm{sto}}  \sum_t g^{\textrm{sto}}_{r,t}  = 0  \hspace{1cm}\perp \hspace{1cm} \lambda_{r} \hspace{1cm} \forall r \label{eq:storconstraint}
\end{equation}

This problem is equivalent to the same problem where we take the optimal shadow prices $\lambda^*_r$, freeze them, take the constraint \eqref{eq:storconstraint} up into the objective function and remove the constraint \eqref{eq:storconstraint} from the problem. The new objective function is then:
\begin{equation}
    \max_{d_{t}, g_{s,t}, G_s,g^{\textrm{dis}}_{r,t},G^{\textrm{dis}}_{r},g^{\textrm{sto}}_{r,t},G^{\textrm{sto}}_{r}}\left[\sum_{t} Vd_{t} -  \sum_s c_s G_s - \sum_{s,t} o_{s} g_{s,t} -\sum_r c^{\textrm{sto}}_r G^{\textrm{sto}}_r -\sum_r c^{\textrm{dis}}_r G^{\textrm{dis}}_r -\sum_r \lambda^*_r\left[ (\eta_r^{\textrm{dis}})^{-1} \sum_t g^{\textrm{dis}}_{r,t} - \eta_r^{\textrm{sto}}  \sum_t g^{\textrm{sto}}_{r,t}\right]  \right]  \label{eq:objst}
\end{equation}
We retain the balance constraint \eqref{eq:balance3}, generation and demand capacity constraints \eqref{eq:generation} and the storage capacity constraints \eqref{eq:storlower2} and \eqref{eq:storupper2}.

The equivalence between the problem with the constraint \eqref{eq:storconstraint} and the problem with the constraint lifted into the objective function is a standard Lagrangian move, which can be checked from the  KKT conditions for each problem.

Now the interpretation of the new objective function: The dispatching storage is behaving exactly like a generator with marginal cost $ (\eta_r^{\textrm{dis}})^{-1} \lambda^*_r $ and capacity $G^{\textrm{dis}}_{r}$, while the charging storage is like a demand with marginal utility $ \eta_r^{\textrm{sto}} \lambda^*_r$  and bid volume $G^{\textrm{sto}}_{r}$.

The price formation will be just like a regular market, and equivalent to the short-term optimisation with frozen capacities $G^*_s,G^{\textrm{sto},*}_{r},G^{\textrm{dis},*}_{r}$ from the long-term model. The objective function for the short-term problem, rearranged to group demand and supply bids, is:
\begin{equation}
    \max_{d_{t},g^{\textrm{sto}}_{r,t}, g_{s,t}, g^{\textrm{dis}}_{r,t}}\left[\sum_{t} Vd_{t} +  \sum_{r,t} \eta_r^{\textrm{sto}}\lambda^*_r g^{\textrm{sto}}_{r,t}   - \sum_{s,t} o_{s} g_{s,t}-\sum_{r,t} (\eta_r^{\textrm{dis}})^{-1}  \lambda^*_r g^{\textrm{dis}}_{r,t} \right]  \label{eq:objst2}
\end{equation}
Since $\eta_r^\circ \leq 1$, the charge bid is seen to be less than the discharge bid (buy low, sell high). If $\eta_r^\circ = 1$ then the bids are the same. This doesn't mean that there is no arbitrage, since the electricity prices $\l_t$ are determined only indirectly by the demand and supply bids.

\section{Full storage model}

Now consider the case where the storage energy capacity $G^{\textrm{ene}}_r$ is also co-optimized, adding to the objective function of the long-term model  $-\sum_r c^{\textrm{ene}}_r G^{\textrm{ene}}_r$. The variable for the state of charge is
$g^{\textrm{ene}}_{r,t}$, and the original state of charge is given as $g^{\textrm{ene}}_{r,0}$. For the bidding strategy it is easier to use the summed expression rather than the difference equation:
\begin{equation}
  g^{\textrm{ene}}_{r,t} = g^{\textrm{ene}}_{r,0} + \sum_{t'=1}^t \left( \eta^{\textrm{sto}}_r g^{\textrm{sto}}_{r,t'} - (\eta^{\textrm{dis}}_r)^{-1} g^{\textrm{dis}}_{r,t'} \right)
\end{equation}

Add the optimization variable $G^{\textrm{ene}}_r$ to the problem for the storage energy capacity and add $-\sum_r c^{\textrm{ene}}_r G^{\textrm{ene}}_r$ to the objective function. Then replace the constraint \eqref{eq:storconstraint} with two extra constraints:
\begin{align}
  -g^{\textrm{ene}}_{r,0} - \sum_{t'=1}^t \left( \eta^{\textrm{sto}}_r g^{\textrm{sto}}_{r,t'} - (\eta^{\textrm{dis}}_r)^{-1} g^{\textrm{dis}}_{r,t'} \right) \leq  0  \hspace{1cm}\perp \hspace{1cm} \ubar{\mu}^{\textrm{ene}}_{r,t} \hspace{1cm} \forall r,t \label{eq:enelower} \\
  g^{\textrm{ene}}_{r,0} + \sum_{t'=1}^t \left( \eta^{\textrm{sto}}_r g^{\textrm{sto}}_{r,t'} - (\eta^{\textrm{dis}}_r)^{-1} g^{\textrm{dis}}_{r,t'} \right) - G^{\textrm{ene}}_r \leq  0  \hspace{1cm}\perp \hspace{1cm} \bar{\mu}^{\textrm{ene}}_{r,t} \hspace{1cm} \forall r,t \label{eq:eneupper}
\end{align}


If we lift these constraints into the objective function, this alters the charging demand bid (coefficient of  $g^{\textrm{sto}}_{r,t}$) to:
\begin{equation}
   \eta^{\textrm{sto}}_r \sum_{t'=t}^T \left( \ubar{\mu}^{\textrm{ene}}_{r,t'} -\bar{\mu}^{\textrm{ene}}_{r,t'}  \right)
\end{equation}
This has a nice interpretation: If at some future point $t'$ the storage is empty, then $\ubar{\mu}^{\textrm{ene}}_{r,t'} \geq 0$  and the bid increases; but if at some time $t'$ the storage is full, then
$\bar{\mu}^{\textrm{ene}}_{r,t'} \geq 0$ and bid decreases.

Similarly for the discharging dispatch bid (coefficient of  $-g^{\textrm{dis}}_{r,t}$, note minus sign of $o_sg_{s,t}$ in objective function) to:
\begin{equation}
   (\eta^{\textrm{dis}}_r)^{-1} \sum_{t'=t}^T \left(\ubar{\mu}^{\textrm{ene}}_{r,t'} - \bar{\mu}^{\textrm{ene}}_{r,t'}  \right)
\end{equation}
If the storage is nearly empty, this increases the price of the dispatch bid; if it's full in future, it decreases the price.

Interesting: like previous section, charging and discharging bids are same up to a constant.

COMPLICATIONS: i) Constraint \eqref{eq:eneupper} also alters the effective capital cost of $G^{\textrm{ene}}_r$? No, this is just the regular stationarity for $G^{\textrm{ene}}_r$, $ c^{\textrm{ene}}_r = \sum_t \bar{\mu}^{\textrm{ene}}_{r,t}$.

ii) Storage may bring down the demand bids into the region where in the long-term model it is cheaper to build additional generation capacity than to shed this load. Not sure. This could mix capital costs into the prices (which storage does anyway). Clarify!


\appendix


\section{Zero profit rule for long-term equilibrium with storage}\label{sec:storage}


Suppose we add storage units $r$ with discharging dispatch $g^{\textrm{dis}}_{r,t}$ and power capacity $G^{\textrm{dis}}_{r}$, storing power $g^{\textrm{sto}}_{r,t}$ and capacity $G^{\textrm{sto}}_{r}$, and state of charge $g^{\textrm{ene}}_{r,t}$ and energy capacity $G^{\textrm{ene}}_{r}$. The efficiency from hour to hour is $\eta^{\textrm{ene}}$ (for losses due to self-discharge), the storing efficiency is $\eta^{\textrm{sto}}$ and the dispatch efficiency is $\eta^{\textrm{dis}}$.

We add to the objective function an additional cost term:
\begin{equation*}
  -\sum_{r,\circ} c^\circ_r G^\circ_r =  -\sum_r c^{\textrm{ene}}_r G^{\textrm{ene}}_r -\sum_r c^{\textrm{sto}}_r G^{\textrm{sto}}_r -\sum_r c^{\textrm{dis}}_r G^{\textrm{dis}}_r
\end{equation*}
where the symbol $\circ$ runs over $\{\textrm{ene},\textrm{sto},\textrm{dis}\}$. We assume no marginal costs for the dispatch.

The demand balancing equation is modified to:
\begin{equation}
   \sum_a d_{a,t} - \sum_s g_{s,t} - \sum_r g^{\textrm{dis}}_{r,t}  + \sum_r g^{\textrm{sto}}_{r,t}  =  0 \hspace{0.34cm}\perp \hspace{0.34cm} \l_t \hspace{0.34cm} \forall t
\end{equation}
The standard capacity constraints apply:
\begin{align}
    -g^\circ_{r,t}\leq 0 \hspace{1cm}\perp \hspace{1cm} \ubar{\mu}^\circ_{r,t} \hspace{1cm} \forall r,t  \label{eq:storlower}\\
    g^\circ_{r,t} - G^\circ_r \leq 0 \hspace{1cm}\perp \hspace{1cm} \bar{\mu}^\circ_{r,t} \hspace{1cm} \forall r,t \label{eq:storupper}
\end{align}
In addition we have the constraint for the consistency of the state of charge between hours according to how much was dispatched or stored:
\begin{equation}
    g^{\textrm{ene}}_{r,t}- \eta^{\textrm{ene}}_r g^{\textrm{ene}}_{r,t-1} - \eta^{\textrm{sto}}_r g^{\textrm{sto}}_{r,t} + (\eta^{\textrm{dis}}_r)^{-1} g^{\textrm{dis}}_{r,t}  =  0 \hspace{0.14cm}\perp \hspace{0.14cm} \l^{\textrm{ene}}_{r,t} \hspace{0.14cm} \forall r,t  \label{eq:storsoc}
\end{equation}
We assume that the state of charge is cyclic $g^{\textrm{ene}}_{r,-1} = g^{\textrm{ene}}_{r,T-1}$.

From KKT stationarity we get:
\begin{align}
\frac{\d \cL}{\d G^\circ_{r}} = 0 &\Rightarrow - c^\circ_r + \sum_t \bar{\m}^\circ_{r,t}  = 0 \\
    \frac{\d \cL}{\d g^{\textrm{dis}}_{r,t}} = 0 &\Rightarrow  \l_t + \ubar{\m}^{\textrm{dis}}_{r,t} - \bar{\m}^{\textrm{dis}}_{r,t} - (\eta^{\textrm{dis}}_r)^{-1} \l^{\textrm{ene}}_{r,t}  = 0 \\
    \frac{\d \cL}{\d g^{\textrm{sto}}_{r,t}} = 0 &\Rightarrow  -\l_t + \ubar{\m}^{\textrm{sto}}_{r,t} - \bar{\m}^{\textrm{sto}}_{r,t} + \eta^{\textrm{sto}}_r \l^{\textrm{ene}}_{r,t}  = 0 \\
    \frac{\d \cL}{\d g^{\textrm{ene}}_{r,t}} = 0 &\Rightarrow   \ubar{\m}^{\textrm{ene}}_{r,t} - \bar{\m}^{\textrm{ene}}_{r,t} -  \l^{\textrm{ene}}_{r,t} + \eta^{\textrm{ene}}_r \l^{\textrm{ene}}_{r,t+1}   = 0
\end{align}

The zero-profit rule for storage proceeds the usual way:
\begin{align}
  \sum_\circ c^\circ_r G^\circ_r & =  \sum_{\circ,t} G^\circ_r\bar{\m}^\circ_{r,t}  =   \sum_{\circ,t} g^\circ_{r,t}\bar{\m}^\circ_{r,t} \nonumber \\
   = & \sum_t \left[ \l_t g^{\textrm{dis}}_{r,t} -(\eta^{\textrm{dis}}_r)^{-1} \l^{\textrm{ene}}_{r,t}  g^{\textrm{dis}}_{r,t}
  -\l_t g^{\textrm{sto}}_{r,t} + \eta^{\textrm{sto}}_r \l^{\textrm{ene}}_{r,t} g^{\textrm{sto}}_{r,t} \right.\nonumber \\
  & \left. \hspace{.5cm}-\l^{\textrm{ene}}_{r,t}g^{\textrm{ene}}_{r,t} + \eta^{\textrm{ene}}_r \l^{\textrm{ene}}_{r,t+1}g^{\textrm{ene}}_{r,t} \right] \nonumber \\
   = & \sum_t \l_t \left[ g^{\textrm{dis}}_{r,t} - g^{\textrm{sto}}_{r,t}  \right] \nonumber \\
   & + \sum_t  \l^{\textrm{ene}}_{r,t} \left[ -(\eta^{\textrm{dis}}_r)^{-1} g^{\textrm{dis}}_{r,t}+ \eta^{\textrm{sto}}_r  g^{\textrm{sto}}_{r,t} -g^{\textrm{ene}}_{r,t} + \eta^{\textrm{ene}}_r g^{\textrm{ene}}_{r,t-1} \right] \nonumber \\
      = & \sum_t \l_t \left[ g^{\textrm{dis}}_{r,t} - g^{\textrm{sto}}_{r,t}  \right]
\end{align}
The first equality is stationarity for $G^\circ_r$; the second is complimentarity for constraint \eqref{eq:storupper}; the third is stationarity for $g^\circ_{r,t}$ and complimentarity for constraint \eqref{eq:storlower}; the fouth rearranges terms and shifts the cyclic sum over $g^{\textrm{ene}}_{r,t}$; the final equality uses the state of charge constraint \eqref{eq:storsoc}.

The final results shows that the storage recovers its capital costs by arbitrage, charging while prices $\l_t$ are low, and discharging while prices are high.

The relation between market value and LCOE of generators in the system are not affected by the introduction of storage (although the optimal capacities may change).

\section{Welfare economics theorem can also be understood via Lagrangian equivalence}

I.e. lifting demand balance constraint into objective, then decomposing into problems for each demand and generator.

\bibliographystyle{elsarticle-num}

\biboptions{sort&compress}
\bibliography{storage_bidding}



\end{document}
